%%%%%%%%%%%%%%%%%%%%%%%%%%%%%%%%%%%%%%%%%%%%%%%%%%%%%%%%%%%%%%%%%%%%%%%%%%%%%%%%%%
\begin{frame}[fragile]\frametitle{}
\begin{center}
{\Large Notes from Naval Ravikant}

\end{center}
\end{frame}




%%%%%%%%%%%%%%%%%%%%%%%%%%%%%%%%%%%%%%%%%%%%%%%%%%%%%%%%%%%
\begin{frame}[fragile]\frametitle{Stress}

	\begin{itemize}
	\item Stress happens when something wants to be in two places at one time. Like iron rod getting pulled from two ends.
	\item Stress is an inability to decide what’s important
	\item You want to find peace from mind. 
	\end{itemize}

\end{frame}

%%%%%%%%%%%%%%%%%%%%%%%%%%%%%%%%%%%%%%%%%%%%%%%%%%%%%%%%%%%
\begin{frame}[fragile]\frametitle{Peace}

	\begin{itemize}
	\item Peace is happiness at rest.
	\item Happiness is Peace in motion.
	\item The ultimate goal is not happiness, even though we use that term a lot. The goal is peace.
	\end{itemize}

\end{frame}

%%%%%%%%%%%%%%%%%%%%%%%%%%%%%%%%%%%%%%%%%%%%%%%%%%%%%%%%%%%
\begin{frame}[fragile]\frametitle{How do you get to peace?}

	\begin{itemize}
	\item Fundamentally, peace is inactivity; it’s a sense that everything is fine.
	\item If everything is fine, you’re not doing any physical or mental activity to change it. 
	\item You’re also not wishing you were doing something to change it, because that creates stress. 
	\end{itemize}

\end{frame}

%%%%%%%%%%%%%%%%%%%%%%%%%%%%%%%%%%%%%%%%%%%%%%%%%%%%%%%%%%%
\begin{frame}[fragile]\frametitle{How do you get to peace?}

	\begin{itemize}
	\item You cannot work toward peace, only understanding
	\item ``The name of God is truth.''
	\item If/once you understand true nature of everything, then you are at Peace.
	\end{itemize}

\end{frame}