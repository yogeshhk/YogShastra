%%%%%%%%%%%%%%%%%%%%%%%%%%%%%%%%%%%%%%%%%%%%%%%%%%%%%%%%%%%%%%%%%%%%%%%%%%%%%%%%%%
\begin{frame}[fragile]\frametitle{}
\begin{center}
{\Large Pratyahar प्रत्याहार}
\end{center}
\end{frame}

%%%%%%%%%%%%%%%%%%%%%%%%%%%%%%%%%%%%%%%%%%%%%%%%%%%%%%%%%%%
\begin{frame}[fragile]\frametitle{Introduction}

Swavishasamprayoge Chittvaswarupanukar eevendrayanang Pratyaharah

स्वविषयासम्प्रयोगे चित्तस्वरुपानुकार इवेन्द्रियाणां प्रत्याहार:||

	\begin{itemize}
	\item Pratyahara  is  withdrawing 
the  senses  or  organs  from 
their  contact  with  the 
objects  in  the  external 
world.  
	\item Sri  Ramakrishna  has 
explained  it  thus  :  the 
moment  an  elephant 
stretches out its trunk to eat 
neighbor’s  garden,  it  gets  a 
blow  from  the  iron  goad  of 
driver 
	\end{itemize}

\end{frame}


