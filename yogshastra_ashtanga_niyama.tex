%%%%%%%%%%%%%%%%%%%%%%%%%%%%%%%%%%%%%%%%%%%%%%%%%%%%%%%%%%%%%%%%%%%%%%%%%%%%%%%%%%
\begin{frame}[fragile]\frametitle{}
\begin{center}
{\Large Niyama नियम}
\end{center}
\end{frame}


%%%%%%%%%%%%%%%%%%%%%%%%%%%%%%%%%%%%%%%%%%%%%%%%%%%%%%%%%%%
\begin{frame}[fragile]\frametitle{Introduction}

Shaucha santosha tapah swadhyayeshwara pranidhanani niyamah

शौच-संतोष-तप:-स्वाध्यायेश्वरप्रणिधानानि नियमा:||

	\begin{itemize}
	\item The  collective  disciplines  of 
Niyama  are  Physical  \&  mental 
purity, contentment, austerity,
self  study  of  holy  books  and 
scriptures and devotion to god. 
\item One  must  dedicate  the  fruits 
and one’s action to god.
	\end{itemize}

\end{frame}

