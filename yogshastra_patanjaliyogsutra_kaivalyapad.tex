%%%%%%%%%%%%%%%%%%%%%%%%%%%%%%%%%%%%%%%%%%%%%%%%%%%%%%%%%%%%%%%%%%%%%%%%%%%%%%%%%%
\begin{frame}[fragile]\frametitle{}
\begin{center}
{\Large Kaivalya Pada कैवल्यपाद}
\end{center}
\end{frame}



%%%%%%%%%%%%%%%%%%%%%%%%%%%%%%%%%%%%%%%%%%%%%%%%%%%%%%%%%%%
\begin{frame}[fragile]\frametitle{Introduction}


	\begin{itemize}
	\item The fourth and final chapter of the Patanjali Yoga Sutras is on moksha मोक्ष, liberation or enlightenment
	\item How the mind is constructed and envelops the inner light of the self. 
	\item It describes how the yogi deals with the overall process and after-effects of enlightenment. 	\item Patanjali outlines his theory of consciousness, how it is constructed and what happens to it when the mind is liberated and the fundamental confusion between the isolated self and a Universal Self.
	\end{itemize}

\tiny{(Ref: Basic Introduction of Patanjali Yoga Sutras – The Best Knowledge for Yogis - Yoga Moha)}

\end{frame}

%%%%%%%%%%%%%%%%%%%%%%%%%%%%%%%%%%%%%%%%%%%%%%%%%%%%%%%%%%%
\begin{frame}[fragile]\frametitle{Kaivalya कैवल्य}


	\begin{itemize}
	\item Kaivalya means ``aloneness'' but it does not refer to isolation from people but it rather refers to the deepest realization, where there is no division between self and other.
	\item We live in an illusion that we are all separate or divided and the very fall of that illusion upon the experiential realization of the oneness is Kaivalya.
	\end{itemize}

\tiny{(Ref: Basic Introduction of Patanjali Yoga Sutras – The Best Knowledge for Yogis - Yoga Moha)}

\end{frame}

%%%%%%%%%%%%%%%%%%%%%%%%%%%%%%%%%%%%%%%%%%%%%%%%%%%%%%%%%%%
\begin{frame}[fragile]\frametitle{Yog}
\begin{sanskrit}
जन्मौषधिमन्त्रतपःसमाधिजाःसिद्धयः॥१॥
\end{sanskrit}

	\begin{itemize}
	\item [HA]: Supernormal Powers Come With Birth Or Are Attained Through Herbs, Incantations, Austerities Or Concentration.
	\item [IT]: The Siddhis are the result of birth, drugs, Mantras austerities or Samadhi.
	\item [VH]: 
	\item [BM]: 
	\item [SS]: Siddhis are born of practices performed in previous births, or by herbs, mantra repetition, asceticism, or by samadhi.
	\item [SP]: The psychic powers may be obtained either by birth, or by means of drugs, or by the power of words, or by the practice of austerities, or by concentration.
	\item [SV]: The Siddhis (powers) are attained by birth, chemical means, power of words, mortification or concentration. 
	\end{itemize}
\end{frame}


%%%%%%%%%%%%%%%%%%%%%%%%%%%%%%%%%%%%%%%%%%%%%%%%%%%%%%%%%%%
\begin{frame}[fragile]\frametitle{Yog}
\begin{sanskrit}
जात्यन्तरपरिणामः प्रकृत्यापूरात्॥२॥
\end{sanskrit}

	\begin{itemize}
	\item [HA]: Takes Place Through The Filling In Of Their Innate Nature.
	\item [IT]: The transformation from one species or kind into another is by the overflow of natural tendencies or potentialities.
	\item [VH]: 
	\item [BM]: 
	\item [SS]: The transformation of one species into another is brought about by the inflow of Nature.
	\item [SP]: The transformation of one species into another is caused by the inflowing of nature.
	\item [SV]: The change into another species is by the filling in of nature. 
	\end{itemize}
\end{frame}

%%%%%%%%%%%%%%%%%%%%%%%%%%%%%%%%%%%%%%%%%%%%%%%%%%%%%%%%%%%
\begin{frame}[fragile]\frametitle{Yog}
\begin{sanskrit}
निमित्तमप्रयोजकं प्रकृतीनां वरणभेदस्तु ततः क्षेत्रिकवत्॥३॥
\end{sanskrit}

	\begin{itemize}
	\item [HA]: Causes Do Not Put Nature Into Motion, Only The Removal Of Obstacles Takes Place Through Them. This Is Like A Farmer Breaking Down the Barrier To Let The Water Flow (The Hindrances Being Removed By The Causes, The Nature Impenetrates By Itself)
	\item [IT]: The incidental cause does not move or stir up the natural tendencies into activity; it merely removes the obstacles, like a farmer (irrigating a field)
	\item [VH]: 
	\item [BM]: 
	\item [SS]: Incidental events do not directly cause natural evolution; they just remove the obstacles as a farmer [removes the obstacles in a water course running to his field].
	\item [SP]: Good or bad deeds are not the direct causes of the transformation. They only act as breakers of the obstacles to natural evolution; just as a farmer breaks down the obstacles in a water course, so that water flows through by its own nature.
	\item [SV]: Good deeds, etc., are not the direct causes in the transformation of nature, but they act as breakers of obstacles to the evolutions of nature, as a farmer breaks the obstacles to the course of water, which then runs down by its own nature.
	\end{itemize}
\end{frame}



%%%%%%%%%%%%%%%%%%%%%%%%%%%%%%%%%%%%%%%%%%%%%%%%%%%%%%%%%%%
\begin{frame}[fragile]\frametitle{Yog}
\begin{sanskrit}
निर्माणचित्तान्यस्मितामात्रात्॥४॥
\end{sanskrit}

	\begin{itemize}
	\item [HA]: All Created Minds Are Constructed From Pure I-sense.
	\item [IT]: Artificially created minds (proceed) from ‘egoism’ alone.
	\item [VH]: 
	\item [BM]: 
	\item [SS]: A Yogi’s egoity alone is the cause of [other artificially] created minds.
	\item [SP]: The ego-sense alone can create minds.
	\item [SV]: From egoism alone proceed the created minds. 
	\end{itemize}
\end{frame}


%%%%%%%%%%%%%%%%%%%%%%%%%%%%%%%%%%%%%%%%%%%%%%%%%%%%%%%%%%%
\begin{frame}[fragile]\frametitle{Yog}
\begin{sanskrit}
प्रवृत्तिभेदे प्रयोजकं चित्तमेकमनेकेषाम्॥५॥
\end{sanskrit}

	\begin{itemize}
	\item [HA]: One (Principal) Mind Directs The Many Created Minds In The Variety Of Their Activities.
	\item [IT]: The one (natural) mind is the director or mover of the many (artificial) minds in their different activities.
	\item [VH]: 
	\item [BM]: 
	\item [SS]: Although the functions in the many created minds may differ, the original mind-stuff of the Yogi is the director of them all.
	\item [SP]: Though the activities of the different created minds are various, the one original mind controls them all.
	\item [SV]: Though the activities of the different created minds are various, the one original mind is the controller of them all. 
	\end{itemize}
\end{frame}


%%%%%%%%%%%%%%%%%%%%%%%%%%%%%%%%%%%%%%%%%%%%%%%%%%%%%%%%%%%
\begin{frame}[fragile]\frametitle{Yog}
\begin{sanskrit}
तत्र ध्यानजमनाशयम्॥६॥
\end{sanskrit}

	\begin{itemize}
	\item [HA]: Of These (Minds With Supernormal Powers) Those Obtained Through Meditation Are Without Any Subliminal Imprints.
	\item [IT]: Of these, the mind born of meditation is free from impressions.
	\item [VH]: 
	\item [BM]: 
	\item [SS]: Only the minds born of meditation [the artificially created ones] are free from karmic impressions.
	\item [SP]: Of the various types of mind, only that which is purified by samadhi is freed from all latent impressions of karma and from all cravings.
	\item [SV]: Among the various Chittas that which is attained by Samadhi is desireless. 
	\end{itemize}
\end{frame}



%%%%%%%%%%%%%%%%%%%%%%%%%%%%%%%%%%%%%%%%%%%%%%%%%%%%%%%%%%%
\begin{frame}[fragile]\frametitle{Yog}
\begin{sanskrit}
कर्माशुक्लाकृष्णं योगिनस्त्रिविधमितरेषाम्॥७॥
\end{sanskrit}

	\begin{itemize}
	\item [HA]: The Actions Of Yogins Are Neither White Nor Black, Whereas The Actions Of Others Are Of Three Kinds.
	\item [IT]: Karmas are neither white nor black (neither good nor bad) in the case of Yogis, they are of three kinds in the case of others.
	\item [VH]: 
	\item [BM]: 
	\item [SS]: The actions of the Yogi are neither white [good] nor black [bad]; but the actions of others are of three kinds: good, bad and mixed.
	\item [SP]: The karma of the yogi is neither white nor black. The karma of others is of three kinds: white, black, or mixed.
	\item [SV]: Works are neither black nor white for the Yogis; for others they are threefold, black, white, and mixed.
	\end{itemize}
\end{frame}


%%%%%%%%%%%%%%%%%%%%%%%%%%%%%%%%%%%%%%%%%%%%%%%%%%%%%%%%%%%
\begin{frame}[fragile]\frametitle{Yog}
\begin{sanskrit}
ततस्तद्विपाकानुगुणानामेवाभिव्यक्तिर्वासनानाम्॥८॥
\end{sanskrit}

	\begin{itemize}
	\item [HA]: Thence (From The Other Three Varieties Of Karma) Are Manifested The Subconscious Impressions Appropriate To Their Consequences.
	\item [IT]: From these only those tendencies are manifested for which the conditions are favourable.
	\item [VH]: 
	\item [BM]: 
	\item [SS]: Of these [actions], only those vasanas (subconscious impressions) for which there are favorable conditions for producing their fruits will manifest in a particular birth.
	\item [SP]: Of the tendencies produced by these three kinds of karma, only those are manifested for which the conditions are favourable.
	\item [SV]: From these threefold works are manifested in each state only those desires (which are) fitting to that state alone. (The others are held in abeyance for the time being.) 
	\end{itemize}
\end{frame}

%%%%%%%%%%%%%%%%%%%%%%%%%%%%%%%%%%%%%%%%%%%%%%%%%%%%%%%%%%%
\begin{frame}[fragile]\frametitle{Yog}
\begin{sanskrit}
जातिदेशकालव्यवहितानामप्यानन्तर्यं स्मृतिसंस्कारयोरेकरूपत्वात्॥९॥
\end{sanskrit}

	\begin{itemize}
	\item [HA]: On Account Of Similarity Between Memory And Corresponding Latent Impressions, The Subconscious Impressions Of Feelings Appear Simultaneously Even When They Are Separated By Birth, Space And Time.
	\item [IT]: There is a relation of cause and effect even though separated bu class, locality and time because memory and impressions are the same in form.
	\item [VH]: 
	\item [BM]: 
	\item [SS]: Although desires are separated from their fulfillments by class, space and time, they have an uninterrupted relationship because the impressions [of desires] and memories of them are identical.
	\item [SP]: Because of our memory of past tendencies, the chain of cause and effect is not broken by change of species, space or time.
	\item [SV]: There is connectiveness in desire, even though separated by speices, space and time, there being identifi-cation of memory and impressions. 
	\end{itemize}
\end{frame}


%%%%%%%%%%%%%%%%%%%%%%%%%%%%%%%%%%%%%%%%%%%%%%%%%%%%%%%%%%%
\begin{frame}[fragile]\frametitle{Yog}
\begin{sanskrit}
तासामनादित्वं चाशिषो नित्यत्वात्॥१०॥
\end{sanskrit}

	\begin{itemize}
	\item [HA]: Desire For Self-Welfare Being Everlasting It Follows That The Subconscious Impression From Which It Arises Must Be Beginningless.
	\item [IT]: And there is no beginning of them, the desire to live being eternal.
	\item [VH]: 
	\item [BM]: 
	\item [SS]: Since the desire to live is eternal, impressions are also beginningless.
	\item [SP]: Since the desire to exist has always been present, our tendencies cannot have had any beginning.
	\item [SV]: Thirst for happiness being eternal desires are without beginning. 
	\end{itemize}
\end{frame}

%%%%%%%%%%%%%%%%%%%%%%%%%%%%%%%%%%%%%%%%%%%%%%%%%%%%%%%%%%%
\begin{frame}[fragile]\frametitle{Yog}
\begin{sanskrit}
हेतुफलाश्रयालम्बनैः संगृहीतत्वादेषामभावे तदभावः॥११॥
\end{sanskrit}

	\begin{itemize}
	\item [HA]: On Account Of Being Held Together By Cause, Result, Refuge And Supporting Object, Vasana Disappears When They Are Absent.
	\item [IT]: Being bound together as cause-effect, substratum-object, they (effects, i.e. Vasanas) disappear on their (cause, i.e. Avidya) disappearance
	\item [VH]: 
	\item [BM]: 
	\item [SS]: The impressions being held together by cause, effect, basis and support, they disappear with the disappearance of these four.
	\item [SP]: Our subconscious tendencies depend upon cause and effect. They have their basis in the mind, and they are stimulated by the sense-objects. If all these are removed, the tendencies are destroyed.
	\item [SV]: Being held together by cause, effect, support, and objects, in the absence of these is its absence. 
	\end{itemize}
\end{frame}


%%%%%%%%%%%%%%%%%%%%%%%%%%%%%%%%%%%%%%%%%%%%%%%%%%%%%%%%%%%
\begin{frame}[fragile]\frametitle{Yog}
\begin{sanskrit}
अतीतानागतं स्वरूपतोऽस्त्यध्वभेदाद्धर्माणाम्॥१२॥
\end{sanskrit}

	\begin{itemize}
	\item [HA]: The Past And The Future Are In Reality Present In Their Fundamental Forms, There Being Only Difference In The Characteristics Of The Forms Taken At Different Times.
	\item [IT]: The past and future exist in their own (real) form. The difference of Dharmas or properties is on account of the difference of paths.
	\item [VH]: 
	\item [BM]: 
	\item [SS]: The past and future exist in the real form of objects which manifest due to differences in the conditions of their characteristics.
	\item [SP]: There is the form and expression we call “past,” and the form and expression we call “future”; both exist within the object, at all times. Form and expression vary according to time—past, present or future.
	\item [SV]: The past and future exist in their own nature, qualities having different ways. 
	\end{itemize}
\end{frame}


%%%%%%%%%%%%%%%%%%%%%%%%%%%%%%%%%%%%%%%%%%%%%%%%%%%%%%%%%%%
\begin{frame}[fragile]\frametitle{Yog}
\begin{sanskrit}
ते व्यक्तसूक्ष्मा गुणात्मानः॥१३॥
\end{sanskrit}

	\begin{itemize}
	\item [HA]: Characteristics, Which Are Present At All Times, Are Manifest And Subtle, And Are Composed Of The Three Gunas.
	\item [IT]: They, whether manifest or unmanifest, are of the nature of Gunas.
	\item [VH]: 
	\item [BM]: 
	\item [SS]: Whether manifested or subtle, these characteristics belong to the nature of the gunas.
	\item [SP]: They are either manifest or subtle, according to the nature of the gunas.
	\item [SV]: They are manifested or fine, being of the nature of the Gunas. 
	\end{itemize}
\end{frame}



%%%%%%%%%%%%%%%%%%%%%%%%%%%%%%%%%%%%%%%%%%%%%%%%%%%%%%%%%%%
\begin{frame}[fragile]\frametitle{Yog}
\begin{sanskrit}
परिणामैकत्वाद्व्स्तुतत्त्वम्॥१४॥ 
\end{sanskrit}

	\begin{itemize}
	\item [HA]: On Account Of The Coordinated Mutation Of The Three Gunas, An Object Appears As A Unit.
	\item [IT]: The essence of the object consists in the uniqueness of transformation (of the Gunas).
	\item [VH]: 
	\item [BM]: 
	\item [SS]: The reality of things is due to the uniformity of the gunas’ transformations.
	\item [SP]: Since the gunas work together within every change of form and expression, there is a unity in all things.
	\item [SV]: The unity in things is from the unity in changes. 
	\end{itemize}
\end{frame}

%%%%%%%%%%%%%%%%%%%%%%%%%%%%%%%%%%%%%%%%%%%%%%%%%%%%%%%%%%%
\begin{frame}[fragile]\frametitle{Yog}
\begin{sanskrit}
वस्तुसाम्ये चित्तभेदात्तयोर्विभक्तः पन्थाः॥१५॥
\end{sanskrit}

	\begin{itemize}
	\item [HA]: In Spite Of The Sameness Of Objects, On Account Of There Being Separate Minds They (The Object And Its Knowledge) Follow Different Paths, That Is Why They Are Entirely Different.
	\item [IT]: The object being the same the difference in the two (the object and its cognition) are due to their (of the minds) separate path.
	\item [VH]: 
	\item [BM]: 
	\item [SS]: Due to differences in various minds, perception of even the same object may vary.
	\item [SP]: The same object is perceived in different ways by different minds. Therefore the mind must be other than the object.
	\item [SV]: Since perception and desire vary with regard to the same object, mind and object are of different nature. 
	\end{itemize}
\end{frame}


%%%%%%%%%%%%%%%%%%%%%%%%%%%%%%%%%%%%%%%%%%%%%%%%%%%%%%%%%%%
\begin{frame}[fragile]\frametitle{Yog}
\begin{sanskrit}
न चैकचित्ततन्त्रं वस्तु तदप्रमाणकं तदा किं स्यात्॥१६॥
\end{sanskrit}

	\begin{itemize}
	\item [HA]: Object Is Not Dependent On One Mind, Because If It Were So, Then What Will Happen When It Is Not Cognised By That Mind.
	\item [IT]: Nor is an object dependent on one mind. What would become of it when not cognized by that mind?
	\item [VH]: 
	\item [BM]: 
	\item [SS]: Nor does an object’s existence depend upon a single mind, for if it did, what would become of that object when that mind did not perceive it?
	\item [SP]: [15A] The object cannot be said to be dependent on the perception of a single mind. For, if this were the case, the object could be said to be non-existent when that single mind was not perceiving it.
	\item [SV]: [note: This sutra does not exist in Vivekananda text. However, the footnote mentions this sutra and provides this translation] The object cannot be said to be dependent on a single mind. There being no proof of its existence, it would then become non-existent. 
	\end{itemize}
\end{frame}


%%%%%%%%%%%%%%%%%%%%%%%%%%%%%%%%%%%%%%%%%%%%%%%%%%%%%%%%%%%
\begin{frame}[fragile]\frametitle{Yog}
\begin{sanskrit}
तदुपरागापेक्षत्वाच्चित्तस्य वस्तु ज्ञाताज्ञातम्॥१७॥
\end{sanskrit}

	\begin{itemize}
	\item [HA]: External Objects Are Known Or Unknown To The Mind According As They Color The Mind.
	\item [IT]: In consequence of the mind being coloured or not coloured by it, an object is known or unknown.
	\item [VH]: 
	\item [BM]: 
	\item [SS]: An object is known of unknown dependent on whether or not the mind gets colored by it.
	\item [SP]: [16] An object is known or unknown, depending upon the moods of the mind.
	\item [SV]: [VN 4.16] Things are known or unknown to the mind, being de-pendent on the colouring which they give to the mind. 
	\end{itemize}
\end{frame}


%%%%%%%%%%%%%%%%%%%%%%%%%%%%%%%%%%%%%%%%%%%%%%%%%%%%%%%%%%%
\begin{frame}[fragile]\frametitle{Yog}
\begin{sanskrit}
सदा ज्ञाताश्चित्तवृत्तयस्तत्प्रभोः पुरुषस्यापरिणामित्वात्॥१८॥
\end{sanskrit}

	\begin{itemize}
	\item [HA]: On Account Of The Immutability Of Purusa Who Is Master Of The Mind, The Modifications Of The Mind Are Always Known Or Manifest.
	\item [IT]: The modifications of the mind are always known to its lord on account of the changelessness of the Purusa.
	\item [VH]: 
	\item [BM]: 
	\item [SS]: Due to His change lessness, changes in the mind-stuff are always known to the Purusha, who is its Lord.
	\item [SP]: [17] Because the Atman, the Lord of the mind, is unchangeable, the mind’s fluctuations are always known to it.
	\item [SV]: [VN 4.17] The states of the mind are always known because the lord of the mind is unchangeable.
	\end{itemize}
\end{frame}



%%%%%%%%%%%%%%%%%%%%%%%%%%%%%%%%%%%%%%%%%%%%%%%%%%%%%%%%%%%
\begin{frame}[fragile]\frametitle{Yog}
\begin{sanskrit}
न तत्स्वाभासं दृश्यत्वात्॥१९॥
\end{sanskrit}

	\begin{itemize}
	\item [HA]: It (Mine) Is Not Self-Illuminating Being An Object (Knowable)
	\item [IT]: Nor is it self-illuminative, for it is perceptible.
	\item [VH]: 
	\item [BM]: 
	\item [SS]: The mind-stuff is not self-luminous because it is an object of perception by the Purusha.
	\item [SP]: [18] The mind is not self-luminous, since it is an object of perception.
	\item [SV]: [VN 4.18] Mind is not self-luminous, being an object. 
	\end{itemize}
\end{frame}

%%%%%%%%%%%%%%%%%%%%%%%%%%%%%%%%%%%%%%%%%%%%%%%%%%%%%%%%%%%
\begin{frame}[fragile]\frametitle{Yog}
\begin{sanskrit}
एकसमये चोभयानवधारणम्॥२०॥
\end{sanskrit}

	\begin{itemize}
	\item [HA]: Besides, Both (The Mind And Its Objects) Cannot Be Cognised Simultaneously.
	\item [IT]: Moreover, it is impossible for it to be of both ways (as perceiver and perceived) at the same time.
	\item [VH]: 
	\item [BM]: 
	\item [SS]: The mind-stuff cannot perceive both subject and object simultaneously [which proves it is not self-luminous].
	\item [SP]: [19] And since it cannot perceive both subject and object simultaneously.
	\item [SV]: [VN 4.19] From its being unable to cognise two things at the same time. 
	\end{itemize}
\end{frame}



%%%%%%%%%%%%%%%%%%%%%%%%%%%%%%%%%%%%%%%%%%%%%%%%%%%%%%%%%%%
\begin{frame}[fragile]\frametitle{Yog}
\begin{sanskrit}
चित्तान्तरदृश्ये बुद्धिबुद्धेरतिप्रसङ्गः स्मृतिसंकरश्च॥२१॥
\end{sanskrit}

	\begin{itemize}
	\item [HA]: If The Mind Were To Be Illumined By Another Mind Then There Will Be Repetition Ad Infinitum Of Illumining Minds And Intermixture Of Memory.
	\item [IT]: If cognition of one mind by another (be postulated) we would have to assume cognition of cognitions and confusion of memories also.
	\item [VH]: 
	\item [BM]: 
	\item [SS]: If the perception of one mind by another mind be postulated, we would have to assume an endless number of them and the result would be confusion of memory.
	\item [SP]: [20] If one postulates a second mind to perceive the first, then one would have to postulate an infinite number of minds; and this would cause confusion of memory.
	\item [SV]: [VN 4.20] Another cognising mind being assumed there will be no end to such assumptions and confusion of memory. 
	\end{itemize}
\end{frame}


%%%%%%%%%%%%%%%%%%%%%%%%%%%%%%%%%%%%%%%%%%%%%%%%%%%%%%%%%%%
\begin{frame}[fragile]\frametitle{Yog}
\begin{sanskrit}
चितेरप्रतिसंक्रमायास्तदाकारापत्तौ स्वबुद्धिसंवेदनम्॥२२॥
\end{sanskrit}

	\begin{itemize}
	\item [HA]: (Though) Untransmissible The Metempiric Consciousness Getting The Likeness Of Buddhi Becomes The Cause Of The Consciousness Of Buddhi.
	\item [IT]: Knowledge of its own nature through self-cognition (is obtained) when consciousness assumes that form in which it does not pass from place to place.
	\item [VH]: 
	\item [BM]: 
	\item [SS]: The consciousness of the Purusha is unchangeable; by getting the reflection of it, the mind-stuff becomes conscious of the Self.
	\item [SP]: [21] The pure consciousness of the Atman is unchangeable. As the reflection of its consciousness falls upon the mind, the mind takes the form of the Atman and appears to be conscious.
	\item [SV]: [VN 4.21] The essence of knowledge (the Purusa) being un-changeable, when the mind takes its form, it becomes conscious.
	\end{itemize}
\end{frame}

%%%%%%%%%%%%%%%%%%%%%%%%%%%%%%%%%%%%%%%%%%%%%%%%%%%%%%%%%%%
\begin{frame}[fragile]\frametitle{Yog}
\begin{sanskrit}
द्रष्टृदृश्योपरक्तं चित्तं सर्वार्थम्॥२३॥
\end{sanskrit}

	\begin{itemize}
	\item [HA]: The Mind-Stuff Being Affected By The Seer And The Seen, Is All-Comprehensive.[IT]: [IT]: The mind coloured by the Knower (i.e., the Purusa) and the Known is all-apprehending.
	\item [VH]: 
	\item [BM]: 
	\item [SS]: The mind-stuff, when colored by both Seer and seen, understands everything.
	\item [SP]: [22] The mind is able to perceive because it reflects both the Atman and the objects of perception.
	\item [SV]: [VN 4.22] Coloured by the seer and the seen the mind is able to understand everything.
	\end{itemize}
\end{frame}


%%%%%%%%%%%%%%%%%%%%%%%%%%%%%%%%%%%%%%%%%%%%%%%%%%%%%%%%%%%
\begin{frame}[fragile]\frametitle{Yog}
\begin{sanskrit}
तदसंख्येयवासनाभिश्र्चित्रमपि परार्थं संहत्यकारित्वात्॥२४॥
\end{sanskrit}

	\begin{itemize}
	\item [HA]: That (The Mind) Though Variegated By Innumerable Subconscious Impressions Exists For Another Since It Acts Conjointly.
	\item [IT]: Through variegated by innumerable Vasanas it (the mind) acts for another (Purusa) for it acts in association.
	\item [VH]: 
	\item [BM]: 
	\item [SS]: Though having countless desires, the mind-stuff exists for the sake of another [the Purusha] because it can act only in association with It.
	\item [SP]: [23] Though the mind has innumerable impressions and desires, it acts only to serve another, the Atman; for, being a compound substance, it cannot act independently, and for its own sake.
	\item [SV]: [VN 4.23] The mind through its innumerable desires acts for another (the Purusa), being combinations. 
	\end{itemize}
\end{frame}


%%%%%%%%%%%%%%%%%%%%%%%%%%%%%%%%%%%%%%%%%%%%%%%%%%%%%%%%%%%
\begin{frame}[fragile]\frametitle{Yog}
\begin{sanskrit}
विशेषदर्शिन आत्मभावभावनाविनिवृत्तिः॥२५॥
\end{sanskrit}

	\begin{itemize}
	\item [HA]: For One Who Has Realised The Distinctive Entity, i.e. Purusa (Mentioned In The Previous Aphorism), Inquiries About The Nature Of His Self Ceases.
	\item [IT]: The cessation (of desire) for dwelling in the consciousness of Atma for one who has seen the distinction.
	\item [VH]: 
	\item [BM]: 
	\item [SS]: To one who sees the distinction between the mind and the Atman, thoughts of mind as the Atman cease forever.
	\item [SP]: [24] The man of discrimination ceases to regard the mind as the Atman.
	\item [SV]: [VN 4.24] For the discriminating the perception of the mind as Atman ceases. 
	\end{itemize}
\end{frame}


%%%%%%%%%%%%%%%%%%%%%%%%%%%%%%%%%%%%%%%%%%%%%%%%%%%%%%%%%%%
\begin{frame}[fragile]\frametitle{Yog}
\begin{sanskrit}
तदा विवेकनिम्नं कैवल्यप्राग्भारं चित्तम्॥२६॥
\end{sanskrit}

	\begin{itemize}
	\item [HA]: Then The Mind Inclines Towards Discriminative Knowledge And Naturally Gravitates Towards The State Of Liberation.
	\item [IT]: Then, verily, the mind is inclined towards discrimination and gravitating towards Kaivalya.
	\item [VH]: 
	\item [BM]: 
	\item [SS]: Then the mind-stuff is inclined toward discrimination and gravitates toward Absoluteness.
	\item [SP]: [25] When the mind is bent on the practice of discrimination, it moves toward liberation.
	\item [SV]: [VN 4.25] Then bent on discriminating the mind attains the previous state of Kaivalya (isolation). 
	\end{itemize}
\end{frame}


%%%%%%%%%%%%%%%%%%%%%%%%%%%%%%%%%%%%%%%%%%%%%%%%%%%%%%%%%%%
\begin{frame}[fragile]\frametitle{Yog}
\begin{sanskrit}
तच्छिद्रेषु प्रत्ययान्तराणि संस्कारेभ्यः॥२७॥
\end{sanskrit}

	\begin{itemize}
	\item [HA]: Through Its Breaches (i.e. Breaks In Discriminative Knowledge) Arise Other Fluctuations Of The Mind Due To (Residual) Latent Impressions.
	\item [IT]: In the intervals arise other Pratyayas from the force of Samskaras.
	\item [VH]: 
	\item [BM]: 
	\item [SS]: In between, distracting thoughts may arise due to past impressions.
	\item [SP]: [26] Distractions due to past impressions may arise if the mind relaxes its discrimination, even a little.
	\item [SV]: [VN 4.26] The thoughts that arise as obstructions to that are from impressions. 
	\end{itemize}
\end{frame}



%%%%%%%%%%%%%%%%%%%%%%%%%%%%%%%%%%%%%%%%%%%%%%%%%%%%%%%%%%%
\begin{frame}[fragile]\frametitle{Yog}
\begin{sanskrit}
हानमेषां क्लेशवदुक्तम्॥२८॥
\end{sanskrit}

	\begin{itemize}
	\item [HA]: It Has Been Said That Their Removal (i.e. Of Fluctuations) Follows The Same Process As The Removal Of Afflictions.
	\item [IT]: Their removal like that of Klesas, as has been described.
	\item [VH]: 
	\item [BM]: 
	\item [SS]: They can be removed, as in the case of the obstacles explained before. [See Book 2, Sutras 1, 2; 10, 11 and 26]
	\item [SP]: [27] They may be overcome in the same manner as the obstacles to enlightenment.
	\item [SV]: [VN 4.27] Their destruction is in the same manner as of ignorance, etc., as said before. 
	\end{itemize}
\end{frame}

%%%%%%%%%%%%%%%%%%%%%%%%%%%%%%%%%%%%%%%%%%%%%%%%%%%%%%%%%%%
\begin{frame}[fragile]\frametitle{Yog}
\begin{sanskrit}
प्रसंख्यानेऽप्यकुसीदस्य सर्वथा विवेकख्यातेर्धर्ममेघः समाधिः॥२९॥
\end{sanskrit}

	\begin{itemize}
	\item [HA]: When One Becomes Disintereested Even In Omniscience One Attains Perpetual Discriminative Enlightenment From Which Ensues The Concentration Known As Dharmamegha (Virtue-Pouring Cloud).
	\item [IT]: In the case of one, who is able to maintain a constant state of Vairagya even towards the most exalted state of enlightenment and to exercise the highest kind of discrimination, follows Charma-Megha-Samadhi.
	\item [VH]: 
	\item [BM]: 
	\item [SS]: He who, due to his perfect discrimination, is totally disinterested even in the highest rewards remains in the constant discriminative discernment, which is called dharmamegha (cloud of dharma) samadhi. [Note: The meaning of dharma includes virtue, justice, law, duty, morality, religion, religious merit, and steadfast decree.]
	\item [SP]: [28] He who remains undistracted even when he is in possession of all the psychic powers, achieves, as the result of perfect discrimination, that samadhi which is called the “cloud of virtue”.
	\item [SV]: [VN 4.28] Even when arriving at the right discriminating knowledge of the senses, he who gives up the fruits, unto him comes as the result of perfect discrimination, the Samadhi called the cloud of virtue. 
	\end{itemize}
\end{frame}


%%%%%%%%%%%%%%%%%%%%%%%%%%%%%%%%%%%%%%%%%%%%%%%%%%%%%%%%%%%
\begin{frame}[fragile]\frametitle{Yog}
\begin{sanskrit}
ततः क्लेशकर्मनिवृत्तिः॥३०॥
\end{sanskrit}

	\begin{itemize}
	\item [HA]: From That Afflictions And Actions Cease.
	\item [IT]: Then follows freedom from Klesas and Karmas.
	\item [VH]: 
	\item [BM]: 
	\item [SS]: From that samadhi all afflictions and karmas cease.
	\item [SP]: [29] Thence come cessation of ignorance, the cause of suffering, and freedom from the power of karma.
	\item [SV]: [VN 4.29] From that comes cessation of pains and works. 
	\end{itemize}
\end{frame}


%%%%%%%%%%%%%%%%%%%%%%%%%%%%%%%%%%%%%%%%%%%%%%%%%%%%%%%%%%%
\begin{frame}[fragile]\frametitle{Yog}
\begin{sanskrit}
तदा सर्वावरणमलापेतस्य ज्ञानस्यानन्त्याज्ज्ञेयमल्पम्॥३१॥
\end{sanskrit}

	\begin{itemize}
	\item [HA]: Then On Account Of The Infinitude Of Knowledge, Freed From The Cover Of All Impurities, The Knowables Appear As Few.
	\item [IT]: Then, in consequence of the removel of all obscuration and impurities, that which can be known (through the mind) is but little in comparison with the infinity of knowledge (obtained in Enlightenment).
	\item [VH]: 
	\item [BM]: 
	\item [SS]: Then all the coverings and impurities of knowledge are totally removed. Because of the infinity of this knowledge, what remains to be known is almost nothing.
	\item [SP]: [30] Then the whole universe, with all its objects of sense-knowledge, becomes as nothing in comparison to that infinite knowledge which is free from all obstructions and impurities.
	\item [SV]: [VN 4.30] Then knowledge, bereft of covering and impurities, becoming infinite, the knowable becomes small. 
	\end{itemize}
\end{frame}



%%%%%%%%%%%%%%%%%%%%%%%%%%%%%%%%%%%%%%%%%%%%%%%%%%%%%%%%%%%
\begin{frame}[fragile]\frametitle{Yog}
\begin{sanskrit}
ततः कृतार्थानां परिणामक्रमसमाप्तिर्गुणानाम्॥३२॥
\end{sanskrit}


	\begin{itemize}
	\item [HA]: After The Emergence Of That (Virtue-Pouring Cloud) The Gunas Having Fulfilled Their Purpose, The Sequence Of Their Mutation Ceases.
	\item [IT]: The three Gunas having fulfilled their object, the process of change (in the Gunas) comes to an end.
	\item [VH]: 
	\item [BM]: 
	\item [SS]: Then the gunas terminate their sequence of transformations because they have fulfilled their purpose.
	\item [SP]: [31] Then the sequence of mutations of the gunas comes to an end, for they have fulfilled their purpose.
	\item [SV]: [VN 4.31] Then are finished the successive transformations of the qualities, they having attained the end. 
	\end{itemize}
\end{frame}


%%%%%%%%%%%%%%%%%%%%%%%%%%%%%%%%%%%%%%%%%%%%%%%%%%%%%%%%%%%
\begin{frame}[fragile]\frametitle{Yog}
\begin{sanskrit}
क्षणप्रतियोगी परिणामापरान्तनिग्रार्ह्यः क्रमः॥३३॥
\end{sanskrit}

	\begin{itemize}
	\item [HA]: What Belongs To The Moments And Is Indicated By The Completion Of A Particular Mutation Is Sequence.
	\item [IT]: The process, corresponding to moments which become apprehensible at the final end of transformation (of the Gunas), is Kramah.
	\item [VH]: 
	\item [BM]: 
	\item [SS]: The sequence [referred to above] means an uninterrupted succession of moments which can be recognized at the end of their transformations.
	\item [SP]: [32] This is the sequence of the mutations which take place at every moment, but which are only perceived at the end of a series.
	\item [SV]: [VN 4.32] The changes that exist in relation to moments, and which are perceived at the other end (at the end of a series) are succession. 
	\end{itemize}
\end{frame}

%%%%%%%%%%%%%%%%%%%%%%%%%%%%%%%%%%%%%%%%%%%%%%%%%%%%%%%%%%%
\begin{frame}[fragile]\frametitle{Yog}
\begin{sanskrit}
पुरुषार्थशून्यानां गुणानां प्रतिप्रसवः कैवल्यं स्वरूपप्रतिष्ठा वा चितिशक्तिरिति॥३४॥
\end{sanskrit}

	\begin{itemize}
	\item [HA]: The State Of The Self In Itself Or Liberation Is Realised When The Gunas (Having Provided For The Experience And Liberation Of Purusa) Are Without Any Objective To Fulfill And Disappear Into Their Causal Substance. In Other Words, It Is Absolute Consciousness Established In Its Own Self.
	\item [IT]: Kaivalya is the state (of Enlightenment) following re-mergence of the Gunas because of their becoming devoid of the object of the Purusa. In this state the Purusa is established in his Real nature which is pure Consciousness.
	\item [VH]: 
	\item [BM]: 
	\item [SS]: Thus, the supreme state of Independence manifests while the gunas reabsorb themselves into Prakriti, having no more purpose to serve the Purusha. Or to look from another angle, the power of pure consciousness settles in its own pure nature.
	\item [SP]: [33] Since the gunas no longer have any purpose to serve for the Atman, they resolve themselves into Prakriti. This is liberation. The Atman shines forth in its own pristine nature, as pure consciousness.
	\item [SV]: [VN 4.33] The resolution in the inverse order of the qualities, berfect of any motive of action for the Purusa, is Kaivalya, or it is the establishment of the power of knowledge in its own nature. 
	\end{itemize}
\end{frame}

