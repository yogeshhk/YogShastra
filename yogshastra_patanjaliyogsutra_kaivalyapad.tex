%%%%%%%%%%%%%%%%%%%%%%%%%%%%%%%%%%%%%%%%%%%%%%%%%%%%%%%%%%%%%%%%%%%%%%%%%%%%%%%%%%
\begin{frame}[fragile]\frametitle{}
\begin{center}
{\Large Kaivalya Pada कैवल्यपाद}
\end{center}
\end{frame}



%%%%%%%%%%%%%%%%%%%%%%%%%%%%%%%%%%%%%%%%%%%%%%%%%%%%%%%%%%%
\begin{frame}[fragile]\frametitle{Introduction}


	\begin{itemize}
	\item The fourth and final chapter of the Patanjali Yoga Sutras is on moksha मोक्ष, liberation or enlightenment
	\item How the mind is constructed and envelops the inner light of the self. 
	\item It describes how the yogi deals with the overall process and after-effects of enlightenment. 	\item Patanjali outlines his theory of consciousness, how it is constructed and what happens to it when the mind is liberated and the fundamental confusion between the isolated self and a Universal Self.
	\end{itemize}

\tiny{(Ref: Basic Introduction of Patanjali Yoga Sutras – The Best Knowledge for Yogis - Yoga Moha)}

\end{frame}

%%%%%%%%%%%%%%%%%%%%%%%%%%%%%%%%%%%%%%%%%%%%%%%%%%%%%%%%%%%
\begin{frame}[fragile]\frametitle{Kaivalya कैवल्य}


	\begin{itemize}
	\item Kaivalya means ``aloneness'' but it does not refer to isolation from people but it rather refers to the deepest realization, where there is no division between self and other.
	\item We live in an illusion that we are all separate or divided and the very fall of that illusion upon the experiential realization of the oneness is Kaivalya.
	\end{itemize}

\tiny{(Ref: Basic Introduction of Patanjali Yoga Sutras – The Best Knowledge for Yogis - Yoga Moha)}

\end{frame}

%%%%%%%%%%%%%%%%%%%%%%%%%%%%%%%%%%%%%%%%%%%%%%%%%%%%%%%%%%%
\begin{frame}[fragile]\frametitle{Yog}
\begin{sanskrit}
जन्मौषधिमन्त्रतपःसमाधिजाःसिद्धयः॥१॥
\end{sanskrit}
	\begin{itemize}
	\item Yog 
	\end{itemize}
\end{frame}


%%%%%%%%%%%%%%%%%%%%%%%%%%%%%%%%%%%%%%%%%%%%%%%%%%%%%%%%%%%
\begin{frame}[fragile]\frametitle{Yog}
\begin{sanskrit}
जात्यन्तरपरिणामः प्रकृत्यापूरात्॥२॥
\end{sanskrit}
	\begin{itemize}
	\item Yog 
	\end{itemize}
\end{frame}

%%%%%%%%%%%%%%%%%%%%%%%%%%%%%%%%%%%%%%%%%%%%%%%%%%%%%%%%%%%
\begin{frame}[fragile]\frametitle{Yog}
\begin{sanskrit}
निमित्तमप्रयोजकं प्रकृतीनां वरणभेदस्तु ततः क्षेत्रिकवत्॥३॥
\end{sanskrit}
	\begin{itemize}
	\item Yog 
	\end{itemize}
\end{frame}



%%%%%%%%%%%%%%%%%%%%%%%%%%%%%%%%%%%%%%%%%%%%%%%%%%%%%%%%%%%
\begin{frame}[fragile]\frametitle{Yog}
\begin{sanskrit}
निर्माणचित्तान्यस्मितामात्रात्॥४॥
\end{sanskrit}
	\begin{itemize}
	\item Yog 
	\end{itemize}
\end{frame}


%%%%%%%%%%%%%%%%%%%%%%%%%%%%%%%%%%%%%%%%%%%%%%%%%%%%%%%%%%%
\begin{frame}[fragile]\frametitle{Yog}
\begin{sanskrit}
प्रवृत्तिभेदे प्रयोजकं चित्तमेकमनेकेषाम्॥५॥
\end{sanskrit}
	\begin{itemize}
	\item Yog 
	\end{itemize}
\end{frame}


%%%%%%%%%%%%%%%%%%%%%%%%%%%%%%%%%%%%%%%%%%%%%%%%%%%%%%%%%%%
\begin{frame}[fragile]\frametitle{Yog}
\begin{sanskrit}
तत्र ध्यानजमनाशयम्॥६॥
\end{sanskrit}
	\begin{itemize}
	\item Yog 
	\end{itemize}
\end{frame}



%%%%%%%%%%%%%%%%%%%%%%%%%%%%%%%%%%%%%%%%%%%%%%%%%%%%%%%%%%%
\begin{frame}[fragile]\frametitle{Yog}
\begin{sanskrit}
कर्माशुक्लाकृष्णं योगिनस्त्रिविधमितरेषाम्॥७॥
\end{sanskrit}
	\begin{itemize}
	\item Yog 
	\end{itemize}
\end{frame}


%%%%%%%%%%%%%%%%%%%%%%%%%%%%%%%%%%%%%%%%%%%%%%%%%%%%%%%%%%%
\begin{frame}[fragile]\frametitle{Yog}
\begin{sanskrit}
ततस्तद्विपाकानुगुणानामेवाभिव्यक्तिर्वासनानाम्॥८॥
\end{sanskrit}
	\begin{itemize}
	\item Yog 
	\end{itemize}
\end{frame}

%%%%%%%%%%%%%%%%%%%%%%%%%%%%%%%%%%%%%%%%%%%%%%%%%%%%%%%%%%%
\begin{frame}[fragile]\frametitle{Yog}
\begin{sanskrit}
जातिदेशकालव्यवहितानामप्यानन्तर्यं स्मृतिसंस्कारयोरेकरूपत्वात्॥९॥
\end{sanskrit}
	\begin{itemize}
	\item Yog 
	\end{itemize}
\end{frame}


%%%%%%%%%%%%%%%%%%%%%%%%%%%%%%%%%%%%%%%%%%%%%%%%%%%%%%%%%%%
\begin{frame}[fragile]\frametitle{Yog}
\begin{sanskrit}
तासामनादित्वं चाशिषो नित्यत्वात्॥१०॥
\end{sanskrit}
	\begin{itemize}
	\item Yog 
	\end{itemize}
\end{frame}

%%%%%%%%%%%%%%%%%%%%%%%%%%%%%%%%%%%%%%%%%%%%%%%%%%%%%%%%%%%
\begin{frame}[fragile]\frametitle{Yog}
\begin{sanskrit}
हेतुफलाश्रयालम्बनैः संगृहीतत्वादेषामभावे तदभावः॥११॥
\end{sanskrit}
	\begin{itemize}
	\item Yog 
	\end{itemize}
\end{frame}


%%%%%%%%%%%%%%%%%%%%%%%%%%%%%%%%%%%%%%%%%%%%%%%%%%%%%%%%%%%
\begin{frame}[fragile]\frametitle{Yog}
\begin{sanskrit}
अतीतानागतं स्वरूपतोऽस्त्यध्वभेदाद्धर्माणाम्॥१२॥
\end{sanskrit}
	\begin{itemize}
	\item Yog 
	\end{itemize}
\end{frame}


%%%%%%%%%%%%%%%%%%%%%%%%%%%%%%%%%%%%%%%%%%%%%%%%%%%%%%%%%%%
\begin{frame}[fragile]\frametitle{Yog}
\begin{sanskrit}
ते व्यक्तसूक्ष्मा गुणात्मानः॥१३॥
\end{sanskrit}
	\begin{itemize}
	\item Yog 
	\end{itemize}
\end{frame}



%%%%%%%%%%%%%%%%%%%%%%%%%%%%%%%%%%%%%%%%%%%%%%%%%%%%%%%%%%%
\begin{frame}[fragile]\frametitle{Yog}
\begin{sanskrit}
परिणामैकत्वाद्व्स्तुतत्त्वम्॥१४॥ 
\end{sanskrit}
	\begin{itemize}
	\item Yog 
	\end{itemize}
\end{frame}

%%%%%%%%%%%%%%%%%%%%%%%%%%%%%%%%%%%%%%%%%%%%%%%%%%%%%%%%%%%
\begin{frame}[fragile]\frametitle{Yog}
\begin{sanskrit}
वस्तुसाम्ये चित्तभेदात्तयोर्विभक्तः पन्थाः॥१५॥
\end{sanskrit}
	\begin{itemize}
	\item Yog 
	\end{itemize}
\end{frame}


%%%%%%%%%%%%%%%%%%%%%%%%%%%%%%%%%%%%%%%%%%%%%%%%%%%%%%%%%%%
\begin{frame}[fragile]\frametitle{Yog}
\begin{sanskrit}
न चैकचित्ततन्त्रं वस्तु तदप्रमाणकं तदा किं स्यात्॥१६॥
\end{sanskrit}
	\begin{itemize}
	\item Yog 
	\end{itemize}
\end{frame}


%%%%%%%%%%%%%%%%%%%%%%%%%%%%%%%%%%%%%%%%%%%%%%%%%%%%%%%%%%%
\begin{frame}[fragile]\frametitle{Yog}
\begin{sanskrit}
तदुपरागापेक्षत्वाच्चित्तस्य वस्तु ज्ञाताज्ञातम्॥१७॥
\end{sanskrit}
	\begin{itemize}
	\item Yog 
	\end{itemize}
\end{frame}


%%%%%%%%%%%%%%%%%%%%%%%%%%%%%%%%%%%%%%%%%%%%%%%%%%%%%%%%%%%
\begin{frame}[fragile]\frametitle{Yog}
\begin{sanskrit}
सदा ज्ञाताश्चित्तवृत्तयस्तत्प्रभोः पुरुषस्यापरिणामित्वात्॥१८॥
\end{sanskrit}
	\begin{itemize}
	\item Yog 
	\end{itemize}
\end{frame}



%%%%%%%%%%%%%%%%%%%%%%%%%%%%%%%%%%%%%%%%%%%%%%%%%%%%%%%%%%%
\begin{frame}[fragile]\frametitle{Yog}
\begin{sanskrit}
न तत्स्वाभासं दृश्यत्वात्॥१९॥
\end{sanskrit}
	\begin{itemize}
	\item Yog 
	\end{itemize}
\end{frame}

%%%%%%%%%%%%%%%%%%%%%%%%%%%%%%%%%%%%%%%%%%%%%%%%%%%%%%%%%%%
\begin{frame}[fragile]\frametitle{Yog}
\begin{sanskrit}
एकसमये चोभयानवधारणम्॥२०॥
\end{sanskrit}
	\begin{itemize}
	\item Yog 
	\end{itemize}
\end{frame}



%%%%%%%%%%%%%%%%%%%%%%%%%%%%%%%%%%%%%%%%%%%%%%%%%%%%%%%%%%%
\begin{frame}[fragile]\frametitle{Yog}
\begin{sanskrit}
चित्तान्तरदृश्ये बुद्धिबुद्धेरतिप्रसङ्गः स्मृतिसंकरश्च॥२१॥
\end{sanskrit}
	\begin{itemize}
	\item Yog 
	\end{itemize}
\end{frame}


%%%%%%%%%%%%%%%%%%%%%%%%%%%%%%%%%%%%%%%%%%%%%%%%%%%%%%%%%%%
\begin{frame}[fragile]\frametitle{Yog}
\begin{sanskrit}
चितेरप्रतिसंक्रमायास्तदाकारापत्तौ स्वबुद्धिसंवेदनम्॥२२॥
\end{sanskrit}
	\begin{itemize}
	\item Yog 
	\end{itemize}
\end{frame}

%%%%%%%%%%%%%%%%%%%%%%%%%%%%%%%%%%%%%%%%%%%%%%%%%%%%%%%%%%%
\begin{frame}[fragile]\frametitle{Yog}
\begin{sanskrit}
द्रष्टृदृश्योपरक्तं चित्तं सर्वार्थम्॥२३॥
\end{sanskrit}
	\begin{itemize}
	\item Yog 
	\end{itemize}
\end{frame}


%%%%%%%%%%%%%%%%%%%%%%%%%%%%%%%%%%%%%%%%%%%%%%%%%%%%%%%%%%%
\begin{frame}[fragile]\frametitle{Yog}
\begin{sanskrit}
तदसंख्येयवासनाभिश्र्चित्रमपि परार्थं संहत्यकारित्वात्॥२४॥
\end{sanskrit}
	\begin{itemize}
	\item Yog 
	\end{itemize}
\end{frame}


%%%%%%%%%%%%%%%%%%%%%%%%%%%%%%%%%%%%%%%%%%%%%%%%%%%%%%%%%%%
\begin{frame}[fragile]\frametitle{Yog}
\begin{sanskrit}
विशेषदर्शिन आत्मभावभावनाविनिवृत्तिः॥२५॥
\end{sanskrit}
	\begin{itemize}
	\item Yog 
	\end{itemize}
\end{frame}


%%%%%%%%%%%%%%%%%%%%%%%%%%%%%%%%%%%%%%%%%%%%%%%%%%%%%%%%%%%
\begin{frame}[fragile]\frametitle{Yog}
\begin{sanskrit}
तदा विवेकनिम्नं कैवल्यप्राग्भारं चित्तम्॥२६॥
\end{sanskrit}
	\begin{itemize}
	\item Yog 
	\end{itemize}
\end{frame}


%%%%%%%%%%%%%%%%%%%%%%%%%%%%%%%%%%%%%%%%%%%%%%%%%%%%%%%%%%%
\begin{frame}[fragile]\frametitle{Yog}
\begin{sanskrit}
तच्छिद्रेषु प्रत्ययान्तराणि संस्कारेभ्यः॥२७॥
\end{sanskrit}
	\begin{itemize}
	\item Yog 
	\end{itemize}
\end{frame}



%%%%%%%%%%%%%%%%%%%%%%%%%%%%%%%%%%%%%%%%%%%%%%%%%%%%%%%%%%%
\begin{frame}[fragile]\frametitle{Yog}
\begin{sanskrit}
हानमेषां क्लेशवदुक्तम्॥२८॥
\end{sanskrit}
	\begin{itemize}
	\item Yog 
	\end{itemize}
\end{frame}

%%%%%%%%%%%%%%%%%%%%%%%%%%%%%%%%%%%%%%%%%%%%%%%%%%%%%%%%%%%
\begin{frame}[fragile]\frametitle{Yog}
\begin{sanskrit}
प्रसंख्यानेऽप्यकुसीदस्य सर्वथा विवेकख्यातेर्धर्ममेघः समाधिः॥२९॥
\end{sanskrit}
	\begin{itemize}
	\item Yog 
	\end{itemize}
\end{frame}


%%%%%%%%%%%%%%%%%%%%%%%%%%%%%%%%%%%%%%%%%%%%%%%%%%%%%%%%%%%
\begin{frame}[fragile]\frametitle{Yog}
\begin{sanskrit}
ततः क्लेशकर्मनिवृत्तिः॥३०॥
\end{sanskrit}
	\begin{itemize}
	\item Yog 
	\end{itemize}
\end{frame}


%%%%%%%%%%%%%%%%%%%%%%%%%%%%%%%%%%%%%%%%%%%%%%%%%%%%%%%%%%%
\begin{frame}[fragile]\frametitle{Yog}
\begin{sanskrit}
तदा सर्वावरणमलापेतस्य ज्ञानस्यानन्त्याज्ज्ञेयमल्पम्॥३१॥
\end{sanskrit}
	\begin{itemize}
	\item Yog 
	\end{itemize}
\end{frame}



%%%%%%%%%%%%%%%%%%%%%%%%%%%%%%%%%%%%%%%%%%%%%%%%%%%%%%%%%%%
\begin{frame}[fragile]\frametitle{Yog}
\begin{sanskrit}
ततः कृतार्थानां परिणामक्रमसमाप्तिर्गुणानाम्॥३२॥
\end{sanskrit}
	\begin{itemize}
	\item Yog 
	\end{itemize}
\end{frame}


%%%%%%%%%%%%%%%%%%%%%%%%%%%%%%%%%%%%%%%%%%%%%%%%%%%%%%%%%%%
\begin{frame}[fragile]\frametitle{Yog}
\begin{sanskrit}
क्षणप्रतियोगी परिणामापरान्तनिग्रार्ह्यः क्रमः॥३३॥
\end{sanskrit}
	\begin{itemize}
	\item Yog 
	\end{itemize}
\end{frame}

%%%%%%%%%%%%%%%%%%%%%%%%%%%%%%%%%%%%%%%%%%%%%%%%%%%%%%%%%%%
\begin{frame}[fragile]\frametitle{Yog}
\begin{sanskrit}
पुरुषार्थशून्यानां गुणानां प्रतिप्रसवः कैवल्यं स्वरूपप्रतिष्ठा वा चितिशक्तिरिति॥३४॥
\end{sanskrit}
	\begin{itemize}
	\item Yog 
	\end{itemize}
\end{frame}

