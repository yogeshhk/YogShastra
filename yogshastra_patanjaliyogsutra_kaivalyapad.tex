%%%%%%%%%%%%%%%%%%%%%%%%%%%%%%%%%%%%%%%%%%%%%%%%%%%%%%%%%%%%%%%%%%%%%%%%%%%%%%%%%%
\begin{frame}[fragile]\frametitle{}
\begin{center}
{\Large Kaivalya Pada कैवल्यपाद}
\end{center}
\end{frame}



%%%%%%%%%%%%%%%%%%%%%%%%%%%%%%%%%%%%%%%%%%%%%%%%%%%%%%%%%%%
\begin{frame}[fragile]\frametitle{Introduction}


	\begin{itemize}
	\item The fourth and final chapter of the Patanjali Yoga Sutras is on moksha मोक्ष, liberation or enlightenment
	\item How the mind is constructed and envelops the inner light of the self. 
	\item It describes how the yogi deals with the overall process and after-effects of enlightenment. 	\item Patanjali outlines his theory of consciousness, how it is constructed and what happens to it when the mind is liberated and the fundamental confusion between the isolated self and a Universal Self.
	\end{itemize}

\tiny{(Ref: Basic Introduction of Patanjali Yoga Sutras – The Best Knowledge for Yogis - Yoga Moha)}

\end{frame}

%%%%%%%%%%%%%%%%%%%%%%%%%%%%%%%%%%%%%%%%%%%%%%%%%%%%%%%%%%%
\begin{frame}[fragile]\frametitle{Kaivalya कैवल्य}


	\begin{itemize}
	\item Kaivalya means ``aloneness'' but it does not refer to isolation from people but it rather refers to the deepest realization, where there is no division between self and other.
	\item We live in an illusion that we are all separate or divided and the very fall of that illusion upon the experiential realization of the oneness is Kaivalya.
	\end{itemize}

\tiny{(Ref: Basic Introduction of Patanjali Yoga Sutras – The Best Knowledge for Yogis - Yoga Moha)}

\end{frame}