%%%%%%%%%%%%%%%%%%%%%%%%%%%%%%%%%%%%%%%%%%%%%%%%%%%%%%%%%%%%%%%%%%%%%%%%%%%%%%%%%%
\begin{frame}[fragile]\frametitle{}
\begin{center}
{\Large Vibhuti Pada विभूतिपाद}
\end{center}
\end{frame}


%%%%%%%%%%%%%%%%%%%%%%%%%%%%%%%%%%%%%%%%%%%%%%%%%%%%%%%%%%%
\begin{frame}[fragile]\frametitle{Introduction}


	\begin{itemize}
	\item The third chapter of the Patanjali Yoga Sutras is about the results, power, and manifestation once the union is achieved. 
		\item It is said, yogis achieve mystical powers (siddhi) due to the regular practice of yoga. However, this chapter notifies yogis that these very same powers can become a hindrance in their path to liberation. 
			\item Furthermore, it warns against the temptations of the eight siddhis or supernatural powers that a yogi can achieve in the higher levels of spiritual development.
	\item Dives deeper into the last three limbs of yoga, which are collectively known as Samyama (संयम).
	\item Patanjali explains how Samyama is used as the finer tool to remove the subtler veils of ignorance, in this chapter.
	\end{itemize}

\tiny{(Ref: Basic Introduction of Patanjali Yoga Sutras – The Best Knowledge for Yogis - Yoga Moha)}

\end{frame}


%%%%%%%%%%%%%%%%%%%%%%%%%%%%%%%%%%%%%%%%%%%%%%%%%%%%%%%%%%%
\begin{frame}[fragile]\frametitle{Dharana}
\begin{sanskrit}
देशबन्धश्चित्तस्य धारणा॥१॥
\end{sanskrit}

	\begin{itemize}
	\item [HA]: Dharana Is The Mind’s (Chitta’s) Fixation On A Particular Point In Space
	\item [IT]: Concentration is the confining of the mind within a limited mental area (0bject of concentration)
	\item [VH]: Dharana-focusing is fixing of the citta- (the energy of) the field of consciousness within a focal point.
	\item [BM]: Concentration is binding the thought in one place.
	\item [SS]: Dharana is the binding of the mind to one place, object or idea.
	\item [SP]: Concentration (dharana) is holding the mind within a center of spiritual consciousness in the body, or fixing it on some divine form, either within the body or outside it.
	\item [SV]: Dharana is holding the mind on to some particular object. 
	\end{itemize}
\end{frame}

%%%%%%%%%%%%%%%%%%%%%%%%%%%%%%%%%%%%%%%%%%%%%%%%%%%%%%%%%%%
\begin{frame}[fragile]\frametitle{Dhyana}
\begin{sanskrit}
तत्र प्रत्ययैकतानता ध्यानम्॥२॥
\end{sanskrit}

	\begin{itemize}
	\item [HA]: In That (Dharana) The Continuous Flow Of Similar Mental Modification Is Called Dhyana Or Meditation.
	\item [IT]: Uninterrupted flow (of the mind) towards the object (chosen for meditation) is contemplation.
	\item [VH]: The single directionality of a pratyaya there (to the chosen focal point) is dhyana-mediation.
	\item [BM]: Meditation is focusing on a single conceptual flow.
	\item [SS]: Dhyana is the continuous flow of cognition toward that object.
	\item [SP]: Meditation (dhyana) is an unbroken flow of thought toward the object of concentration.
	\item [SV]: An unbroken flow of knowledge to that object is Dhyana. 
	\end{itemize}
\end{frame}



%%%%%%%%%%%%%%%%%%%%%%%%%%%%%%%%%%%%%%%%%%%%%%%%%%%%%%%%%%%
\begin{frame}[fragile]\frametitle{Samadhi}
\begin{sanskrit}
तदेवार्थमात्रनिर्भासं स्वरूपशून्यमिव समाधिः॥३॥
\end{sanskrit}

	\begin{itemize}
	\item [HA]: When The Object Of Meditation Only Shines Forth In The Mind, As Though Devoid Of The Thought Of Even The Self (Who Is Meditation) That State Is Called Samadhi Or Concentration.
	\item [IT]: The same (contemplation) when there is consciousness only of the object of meditation and not of itself (the mind) is Samadhi.
	\item [VH]: That (citta) specifically, reflecting as the object alone, as if empty of its own form, is samadhi-cognitive absorption.
	\item [BM]: Pure contemplation is meditation that illumines the object alone, as if the subject were devoid of intrinsic form.
	\item [SS]: Samadhi is the same meditation when there is the shining of the object alone, as if devoid of form.
	\item [SP]: When, in meditation, the true nature of the object shines forth, not distorted by the mind of the perceiver, that is absorption (samadhi).
	\item [SV]: When that, giving up all forms, reflects only the meaning, it is Samadhi. 
	\end{itemize}
\end{frame}


%%%%%%%%%%%%%%%%%%%%%%%%%%%%%%%%%%%%%%%%%%%%%%%%%%%%%%%%%%%
\begin{frame}[fragile]\frametitle{Yog}
\begin{sanskrit}
त्रयमेकत्र संयमः॥४॥
\end{sanskrit}

	\begin{itemize}
	\item [HA]: The Three Together On The Same Object Is Called Samyama.
	\item [IT]: These three taken together constitute Samyama.
	\item [VH]: The group of three (dharana, dhyana, and Samadhi) together as one is sanyama – the perfect regulation of citta.
	\item [BM]: Concentration, meditation, and pure contemplation focused on a single object constitute perfect discipline.
	\item [SS]: The practice of these three [dharana, dhyana and samadhi] upon one object is called samyama.
	\item [SP]: When these three—concentration, meditation and absorption—are brought to bear upon one subject, they are called samyama.
	\item [SV]: (These) three (when practised) in regard to one object is Samyama. 
	\end{itemize}
\end{frame}


%%%%%%%%%%%%%%%%%%%%%%%%%%%%%%%%%%%%%%%%%%%%%%%%%%%%%%%%%%%
\begin{frame}[fragile]\frametitle{Samyama}
\begin{sanskrit}
तज्जयात्प्रज्ञालोकः॥५॥
\end{sanskrit}

	\begin{itemize}
	\item [HA]: By Mastering That, The Light Of Knowledge (Prajna) Dawns.
	\item [IT]: By mastering it (Samyama) the light of higher consciousness.
	\item [VH]: Owing to the success of that (sanyama), the brilliance of prajna-insight.
	\item [BM]: The light of wisdom comes from mastery of perfect discipline.
	\item [SS]: By mastery of samyama comes the light of knowledge.
	\item [SP]: Through mastery of samyama comes the light of knowledge.
	\item [SV]: By the conquest of that comes light of knowledge. 
	\end{itemize}
\end{frame}



%%%%%%%%%%%%%%%%%%%%%%%%%%%%%%%%%%%%%%%%%%%%%%%%%%%%%%%%%%%
\begin{frame}[fragile]\frametitle{Stages}
\begin{sanskrit}
तस्य भूमिषु विनियोगः॥६॥
\end{sanskrit}

	\begin{itemize}
	\item [HA]: It (Samyama) Is To Be Applied To The Stages (Of Practice)
	\item [IT]: Its (of Samyama) use by stages.
	\item [VH]: Its (sanyama’s) application is in stages.
	\item [BM]: The practice of perfect discipline is achieved in stages.
	\item [SS]: Its practice is to be accomplished in stages.
	\item [SP]: It must be applied stage by stage.
	\item [SV]: That should be employed in stages. 
	\end{itemize}
\end{frame}


%%%%%%%%%%%%%%%%%%%%%%%%%%%%%%%%%%%%%%%%%%%%%%%%%%%%%%%%%%%
\begin{frame}[fragile]\frametitle{Intimate Practices}
\begin{sanskrit}
त्रयमन्तरङ्गं पूर्वेभ्यः॥७॥
\end{sanskrit}

	\begin{itemize}
	\item [HA]: These Three Are More Intimate Practices Than The Previously Mentioned Ones.
	\item [IT]: The three are internal in relation to the preceding ones.
	\item [VH]: The group of three (dharana, dhyana, and Samadhi) is the inner limb(s) distinct from the previous (five limbs of yoga)
	\item [BM]: In contrast with the prior limbs of yoga, the final triad is internal.
	\item [SS]: These three [dharana, dhyana and Samadhi] are more internal than the preceding five limbs.
	\item [SP]: These three are more direct aids to experience than the five limbs previously described.
	\item [SV]: These three are nearer than those that precede. 
	\end{itemize}
\end{frame}


%%%%%%%%%%%%%%%%%%%%%%%%%%%%%%%%%%%%%%%%%%%%%%%%%%%%%%%%%%%
\begin{frame}[fragile]\frametitle{Samadhi}
\begin{sanskrit}
तदपि बहिरङ्गं निर्बीजस्य॥८॥
\end{sanskrit}

	\begin{itemize}
	\item [HA]: That Also Is (To Be Regarded As) External In Respect Of Nirvija Or Seedless Concentration.
	\item [IT]: Even that (Sabija Samadhi) is external to the Seedless (Nirbija Samhadhi)
	\item [VH]: That (inner limb) however is an external limb of nirbija-the seedless (Samadhi-cognitive absorption)
	\item [BM]: Yet it is only an external limb of seedless contemplation.
	\item [SS]: Even these three are external to the seedless samadhi.
	\item [SP]: But even these are not direct aids to the seedless samadhi.
	\item [SV]: But even they are external to the seedless (Samadhi). 
	\end{itemize}
\end{frame}


%%%%%%%%%%%%%%%%%%%%%%%%%%%%%%%%%%%%%%%%%%%%%%%%%%%%%%%%%%%
\begin{frame}[fragile]\frametitle{Suppression}
\begin{sanskrit}
व्युत्थाननिरोधसंस्कारयोरभिभवप्रादुर्भावौ निरोधक्षणचित्तान्वयो निरोधपरिणामः॥९॥
\end{sanskrit}

	\begin{itemize}
	\item [HA]: Suppression Of The Latencies Of Fluctuation And Appearance Of The Latencies Of Arrested State Taking Place At Every Moment Of Blankness Of The Arrested State In The Same Mind, Is The Mutation Of The Arrested State Of Mind.
	\item [IT]: Nirodha Parinama is that transformation of the mind in which it becomes progressively permeated by that condition of Nirodha which intervenes momentarily between an impression which is disappearing and the impression which is taking place.
	\item [VH]: The submergence of the sanskara-subliminal activator of externalization and the emergence of the sanskara of nirodha- the act of ending (citta-vrtti) is the nirodha-parinama (nirodha-transformation) which is connected to citta- the energy field of consciousness at the moment of nirodha.
	\item [BM]: The transformation of thought leading toward its own cessation is accompanied by moments of cessation, when subliminal impression of mental distraction are overcome and those of cessation emerge in their place.
	\item [SS]: The impressions which normally arise are made to disappear by the appearance of suppressive efforts, which in turn create mental modifications. The moment of conjunction of mind and new modification is nirodha parinama.
	\item [SP]: When the vision of the lower samadhi is suppressed by an act of conscious control, so that there are no longer any thoughts or visions in the mind, that is the achievement of control of the thought-waves of the mind.
	\item [SV]: By the suppression of the disturbed modifications of the mind, and by the rise of modifications of control, the mind is said to attain the controlling modifications —following the controlling powers of the mind. 
	\end{itemize}
\end{frame}

%%%%%%%%%%%%%%%%%%%%%%%%%%%%%%%%%%%%%%%%%%%%%%%%%%%%%%%%%%%
\begin{frame}[fragile]\frametitle{Tranquil Mind}
\begin{sanskrit}
तस्य प्रशान्तवाहिता संस्कारात्॥१०॥
\end{sanskrit}

	\begin{itemize}
	\item [HA]: Continuity Of The Tranquil Mind (In An Arrested State) Is Ensured By The Latent Impressions.
	\item [IT]: It’s flow becomes tranquil by repeated impression.
	\item [VH]: The calm flow of that (nirodha-parinama-tranformation) occurs due to the sanskara-subliminal activator (or nirodha)
	\item [BM]: From subliminal impression of these moments, the flow of tranquility is constant.
	\item [SS]: The flow of nirodha parinama becomes steady through habit.
	\item [SP]: When this suppression of thought-waves becomes continuous, the mind’s flow is calm.
	\item [SV]: Its flow becomes steady by habit. 
	\end{itemize}
\end{frame}



%%%%%%%%%%%%%%%%%%%%%%%%%%%%%%%%%%%%%%%%%%%%%%%%%%%%%%%%%%%
\begin{frame}[fragile]\frametitle{Distractness}
\begin{sanskrit}
सर्वार्थतैकाग्रतयोः क्षयोदयौ चित्तस्य समाधिपरिणामः॥११॥
\end{sanskrit}

	\begin{itemize}
	\item [HA]: Diminution Of Attention To All And Sundry And Development Of One Pointedness Is Called Samadhi-Parinama Or Mutation Of The Concentrative Mind.
	\item [IT]: Samadhi transformation is the (gradual) setting of the distraction and simultaneous rising of one-pointedness.
	\item [VH]: The disappearance of the sarvatha- all-objectiveness and the arising of ekagrata-one-pointedness is Samadhi-parinama (Samadhi-tranformation) of citta-the energy friend of consciousness.
	\item [BM]: The transformation of thought towards pure contemplation occurs when concern for all external objects declines and psychic focus arises.
	\item [SS]: When there is decline in distractness and appearance of one-pointedness, then comes samadhi parinama (development in Samadhi)
	\item [SP]: When all mental distractions disappear and the mind becomes one-pointed, it enters the state called samadhi.
	\item [SV]: Taking in all sorts of objects and concentrating upon one object, these two powers being destroyed and manifested 
	\end{itemize}
\end{frame}


%%%%%%%%%%%%%%%%%%%%%%%%%%%%%%%%%%%%%%%%%%%%%%%%%%%%%%%%%%%
\begin{frame}[fragile]\frametitle{Concentration}
\begin{sanskrit}
ततः पुनः शान्तोदितौ तुल्यप्रत्ययौ चित्तस्यैकाग्रतापरिणामः॥१२॥
\end{sanskrit}

	\begin{itemize}
	\item [HA]: There (In Samadhi) Again (In The State Of Concentration) The Past And The Present Modifications Being Similar It Is Ekagrata-Parinama Or Mutation Of The Stabilised State Of Mind.
	\item [IT]: Then, again, the condition of the mind in which the ‘object’ (in the mind) which is always exactly similar to the ‘object’ which rises (in the next moment) is called Ekagrata Parinama.
	\item [VH]: Then again, when the santa-quieted and the udita-arisen are the same pratyaya, there is ekagrata-parinima-one-pointedness transformation of the citta-the field.
	\item [BM]: The transformation of thought towards psychic focus occurs when a concept is equally at rest or arising.
	\item [SS]: Then again, when subsiding past and rsing present images are identical, there is ekagrata parinam (one pointedness).
	\item [SP]: The mind becomes one-pointed when similar thoughtwaves arise in succession without any gaps between them.
	\item [SV]: The one-pointedness of the Chitta is when it grasps in one, the past and present.
	\end{itemize}
\end{frame}


%%%%%%%%%%%%%%%%%%%%%%%%%%%%%%%%%%%%%%%%%%%%%%%%%%%%%%%%%%%
\begin{frame}[fragile]\frametitle{Changes}
\begin{sanskrit}
एतेन भूतेन्द्रियेषु धर्मलक्षणावस्थापरिणामा व्याख्याताः॥१३॥
\end{sanskrit}

	\begin{itemize}
	\item [HA]: By These Are Explained The Three Changes, Viz. Of Essential Attributes Or Characteristics, Of Temporal Characters, And Of States Of The Bhutas And The Indriyas (i.e. All The Knowable Phenomena)
	\item [IT]: By this (by what has been said in the last four Sutras) the property, character and the sense-organs are also explained.
	\item [VH]: By this are explained the parinima-transformation of dharma-characteristic form, laksana-potential change and avastha-condition in regard to the bhuta-element and indriya-sense organs.
	\item [BM]: By extension, these transformation of thought explain the transformation of nature’s properties, characteristics, and conditions, which occur in material elements and sense organs.
	\item [SS]: By this [what has been said in the preceding Sutras], the transformation of the visible characteristics, time factors and conditions of elements and senses are also described.
	\item [SP]: In this state, it passes beyond the three kinds of changes which take place in subtle or gross matter, and in the organs: change of form, change of time and change of condition.
	\item [SV]: By this is explained the threefold transformations of form, time and state, in fine or gross matter, and in the organs. 
	\end{itemize}
\end{frame}

%%%%%%%%%%%%%%%%%%%%%%%%%%%%%%%%%%%%%%%%%%%%%%%%%%%%%%%%%%%
\begin{frame}[fragile]\frametitle{Yog}
\begin{sanskrit}
शान्तोदिताव्यपदेश्यधर्मानुपाती धर्मी॥१४॥
\end{sanskrit}

	\begin{itemize}
	\item [HA]: That Which Continues Its Existence All Through The Varying Characteristics, Namely The Quiscent, i.e. Past, The Uprisen, i.e. Present Or Unmanifest (But Remaining As Potent Force) i.e. Future, Is The Substratum (Or Object Characterised).
	\item [IT]: The substratum is that in which the properties – latent, active or unmanifest – inhere.
	\item [VH]: The form substratum (dharmi) conforms to the characteristic form, which may be Quieted, arisen, and indistinguishable (past, present, and future).
	\item [BM]: The substratum underlying the essential properties of material nature endures whether these properties are at rest, arising, or unmanifest.
	\item [SS]: It is the substratum (Prakriti) that by nature goes through latent, uprising and unmanifest phases.
	\item [SP]: A compound object has attributes and is subject to change, either past, present or yet to be manifested.
	\item [SV]: That which is acted upon by transformations, either past, present or yet to be manifested, is the qualified. 
	\end{itemize}
\end{frame}

%%%%%%%%%%%%%%%%%%%%%%%%%%%%%%%%%%%%%%%%%%%%%%%%%%%%%%%%%%%
\begin{frame}[fragile]\frametitle{Yog}
\begin{sanskrit}
क्रमान्यत्वं परिणामान्यत्वे हेतुः॥१५॥
\end{sanskrit}

	\begin{itemize}
	\item [HA]: Change Of Sequence (Of Characteristics) Is Cause Of Mutative Differences.
	\item [IT]: The cause of the difference in transformation is the difference in the underlying process.
	\item [VH]: The separateness of the krama-sequential progression (of each citta-field) is the reason for the separateness of parinima-transformations.
	\item [BM]: Variation in the sequence of properties cause difference in the transformations of nature.
	\item [SS]: The succession of these different phases is the cause of the difference in stages of evolution.
	\item [SP]: The succession of these changes is the cause of manifold evolution.
	\item [SV]: The succession of changes is the cause of manifold evolution. 
	\end{itemize}
\end{frame}


%%%%%%%%%%%%%%%%%%%%%%%%%%%%%%%%%%%%%%%%%%%%%%%%%%%%%%%%%%%
\begin{frame}[fragile]\frametitle{Samyama}
\begin{sanskrit}
परिणामत्रयसंयमासदतीतानागतज्ञानम्॥१६॥
\end{sanskrit}

	\begin{itemize}
	\item [HA]: Knowledge Of The Past And The Furute Can Be Derived Through Samyama On The Three Parinamas (Changes)
	\item [IT]: By performing Samyama on the three kinds of transformation (Nirodha, Samadhi and Ekagrata) knowledge of the past and future.
	\item [VH]: Due to the sanyama (perfect regulation of citta by dharma, dyana, samahdi) on the three transformations (dharma-characteristi, laksana-potential change, avastha-condition) there arises knowledge of the past and future.
	\item [BM]: Knowledge of the past and future comes form perfect discipline of the three transformations of thought.
	\item [SS]: By practicing samyama on the three stages of evolution comes knowledge of past and future.
	\item [SP]: By making samyama on the three kinds of changes, one obtains knowledge of past and the future.
	\item [SV]: By making Samyama on the three sorts of changes comes the knowledge of past and future. 
	\end{itemize}
\end{frame}

%%%%%%%%%%%%%%%%%%%%%%%%%%%%%%%%%%%%%%%%%%%%%%%%%%%%%%%%%%%
\begin{frame}[fragile]\frametitle{Samyama}
\begin{sanskrit}
शब्दार्थप्रत्ययानामितरेतराध्यासात् संकरस्तत्प्रविभागसंयमात् सर्वभूतरुतज्ञानम्॥१७॥
\end{sanskrit}

	\begin{itemize}
	\item [HA]: Word, Object Implied And The Idea Thereof Overlapping, Produce One Unified Impression. If Samyama Is Practiced On Each Separately, Knowledge Of The Meaning Of The Sounds Produces By All Beings Can Be Acquired.
	\item [IT]: The sound, the meaning (behind it) and the idea (which is present in the mind at the time) are present together in a confused state. By performing Samyama (on the sound) they are resolved and there arises comprehension of the meaning of sounds uttered by any living being.
	\item [VH]: The confusion of words, meanings, and pratyaya is due to the super-imposition of one upon the other. By sanyama (perfect regulation of citta) on the inherent distinctness of these, there arises knowledge of the sound of all beings.
	\item [BM]: Confusion arises from erroneously identifying words, objects, and ideas with one another; knowledge of the cries of all creatures comes through perfect discipline of the distinctions between them
	\item [SS]: A word, its meaning, and the idea behind it are normally confused because of superimposition upon one and another. By samyam on the word [or sound] produced by any being, knowledge of its meaning is obtained.
	\item [SP]: By making samyama on the sound of a word, one’s perception of its meaning, and one’s reaction to it-three things which are ordinarily confused—one obtains understanding of all sounds uttered by living beings.
	\item [SV]: By making Samyama on word, meaning, and knowledge, which are ordinarily confused, comes the knowledge of all animal sounds. 
	\end{itemize}
\end{frame}

%%%%%%%%%%%%%%%%%%%%%%%%%%%%%%%%%%%%%%%%%%%%%%%%%%%%%%%%%%%
\begin{frame}[fragile]\frametitle{Realisation}
\begin{sanskrit}
संस्कारसाक्षत्करणात् पूर्वजातिज्ञानम्॥१८॥
\end{sanskrit}

	\begin{itemize}
	\item [HA]: By The Realisation Of Latent Impression, Knowledge Of Previous Births Is Acquired.
	\item [IT]: By direct perceptions of the impression a knowledge of previous births.
	\item [VH]: By direct perception of sanskara-subliminal activators, knowledge of previous births.
	\item [BM]: Through direct perception of one’s subliminal impression, one has knowledge of former births.
	\item [SS]: By direct perception, through samyama, of one’s mental impressions, knowledge of past births is obtained.
	\item [SP]: By making samyama on previous thought-waves, one obtains knowledge of one’s past lives.
	\item [SV]: By perceiving the impressions, knowledge of past life. 
	\end{itemize}
\end{frame}

%%%%%%%%%%%%%%%%%%%%%%%%%%%%%%%%%%%%%%%%%%%%%%%%%%%%%%%%%%%
\begin{frame}[fragile]\frametitle{Pratyaya}
\begin{sanskrit}
प्रत्ययस्य परचित्तज्ञानम्॥१९॥
\end{sanskrit}

	\begin{itemize}
	\item [HA]: (By Practicing Samyama) On Notions, Knowledge Of Other Minds Is Developed.
	\item [IT]: (By direct perception through Samyama) of the image occupying the mind, knowledge of the mind of others.
	\item [VH]: (By direct perception) of a pratyaya- knowledge of the citta of another.
	\item [BM]: Through direct perception of cognitive process, one has knowledge of the thoughts of others.
	\item [SS]: By samyama on the distinguishing signs of others’ bodies, knowledge of their mental images is obtained.
	\item [SP]: By making samyama on the distinguishing marks of another man’s body, one obtains knowledge of the nature of his mind.
	\item [SV]: By making Samyama on the signs in another’s both knowledge of that mind comes. 
	\end{itemize}
\end{frame}


%%%%%%%%%%%%%%%%%%%%%%%%%%%%%%%%%%%%%%%%%%%%%%%%%%%%%%%%%%%
\begin{frame}[fragile]\frametitle{Samyama}
\begin{sanskrit}
न च तत् सालम्बनं तस्याविषयीभूतत्वात्॥२०॥
\end{sanskrit}

	\begin{itemize}
	\item [HA]: The Prop (Or Basis) Of The Notion Does Not Get Known Because That Is Not The Object Of The (Yogin’s ) Observation.
	\item [IT]: But not also of other mental factors which support the mental image for that is not the object (of Samyama).
	\item [VH]: And it is not that (citta) together with its supporting object. Due to its (citta’s) nature of being that which has not object.
	\item [BM]: But this does not involve knowledge of the underlying object of thought since that is not the object of one’s perception.
	\item [SS]: But this does not include support in the person’s mind [such as the motive behind the thought, etc.], as that is not the object of samyama.
	\item [SP]: But not of its contents, because that is not the object of the samyama.
	\item [SV]: But not its contents, that not being the object of the Samyama. 
	\end{itemize}
\end{frame}

%%%%%%%%%%%%%%%%%%%%%%%%%%%%%%%%%%%%%%%%%%%%%%%%%%%%%%%%%%%
\begin{frame}[fragile]\frametitle{Yog}
\begin{sanskrit}
कायरूपसंयमात् तद्ग्राह्यशक्तिस्तम्भे चक्षुःप्रकाशासंप्रयोगेऽन्तर्धानम्॥२१॥
\end{sanskrit}

	\begin{itemize}
	\item [HA]: When Perceptibility Of The Body Is Suppressed By Practicing Samyama On The Visual Character, Disappearance Of The Body Is Effected Through It’s Getting Beyond The Sphere Of Perception Of The Eye.
	\item [IT]: By performing Samyama on Rupa (one of the five Tanmatras), on suspension of the receptive power, the contact between the eye (of the observer) and light (from the body) is broken and the whole body becomes invisible.
	\item [VH]: By sanyama on form of the body, while suspending its ability to be seen, that is, the disconnecting of light to the eye – there arises invisibility (placement within).
	\item [BM]: From perfect discipline of the body’s own form, one can become invisible by paralyzing the power to perceive one’s body and blocking the contact of light from one’s eyes.
	\item [SS]: By samyama on the form of one’s body, [and by] checking the power of perception by intercepting light from the eyes of the observer, the body becomes invisible.
	\item [SP]: If one makes samyama on the form of one’s body obstructing its perceptibility and separating its power of manifestation from the eyes of the beholder, then one’s body becomes invisible.
	\item [SV]: By making Samyama on the form of the body the power of perceiving forms being obstructed, the power of manifestation in the eye being separated, the Yogi’s body becomes unseen. 
	\end{itemize}
\end{frame}

%%%%%%%%%%%%%%%%%%%%%%%%%%%%%%%%%%%%%%%%%%%%%%%%%%%%%%%%%%%
\begin{frame}[fragile]\frametitle{Yog}
\begin{sanskrit}
एतेन शब्दाद्यन्तर्धानमुक्तम् ॥२२॥
\end{sanskrit}

	\begin{itemize}
	\item [SS]: do. This changes the numbering of the following sutras; Taimini and [SS]: are numbered one ahead. I’ve included their text with the rest of the translators for practicality in comparing the meanings, and kept their self-referential numbers intact as an aside.)
	\item [IT]: From the above can be understood the disappearance of sound, etc.
	\item [SS]: In the same way, the disappearance of sound [touch, taste, smell, etc] is explained.
	\item [SP]: Thus, also, its sounds cease to be heard.
	\item [SV]: By this the disappearance or concealment of words which are being spoken is also explained.
	\end{itemize}
\end{frame}

%%%%%%%%%%%%%%%%%%%%%%%%%%%%%%%%%%%%%%%%%%%%%%%%%%%%%%%%%%%
\begin{frame}[fragile]\frametitle{Yog}
\begin{sanskrit}
सोपक्रमं निरुपक्रमं च कर्म तत्संयमादपरान्तज्ञानमरिष्टेभ्यो वा॥२२॥
\end{sanskrit}

	\begin{itemize}
	\item [HA]: Karma Is Either Fast Or Slow In Fructifying. By Practicing Samayama On Karma Or On Portents, Fore-Knowledge Of Death Can Be Acquired.
	\item Karma is of two kinds: active and dormant; by performing Samyama on them (is gained) knowledge of the time of death; also by (performing Samyama) on portents.
	\item [VH]: Karma is either sopakrama-with the advance of krama-sequential progression (fast in fruition) or nirupakrama-against the advance of krama (slow in fruition). The knowledge of time of death may be known by sanyama upon that or by signs.
	\item [BM]: From perfect discipline of the immediate and remote effects of action, or of omens, one has foreknowledge of death.
	\item [SS]: (III.23): Karmas are of two kinds: quickly manifesting and slowly manifesting. By samyama on them, or on the portents of death, the knowledge of the time of death is obtained.
	\item [SP]: By making samyama on two kinds of karma-which will soon bear fruit and that which will not fruit until later—or by recognizing the portents of death, a yogi may know the exact time of his separation from the body.
	\item [SV]: Karma is of two kinds, soon to be fructified, and late to be fructified. By making Samyama on that, or by the signs called Aristha, portents, the Yogis know the exact time of separation from their bodies. 
	\end{itemize}
\end{frame}


%%%%%%%%%%%%%%%%%%%%%%%%%%%%%%%%%%%%%%%%%%%%%%%%%%%%%%%%%%%
\begin{frame}[fragile]\frametitle{Fructifying}
\begin{sanskrit}
मैत्र्यादिषु बलानि॥२३॥
\end{sanskrit}

	\begin{itemize}
	\item [HA]: Through Samyama On Friendliness (Amity) And Other Similar Virtues, Strength Is Obtained Therein.
	\item [IT]: (24): (By performing Smayama) on friendliness etc (comes) strength (of that quality)
	\item [VH]: (By sanyama) on friendship, etc. – strengths (I.33)
	\item [BM]: From perfect discipline of friendship, compassion, joy, and impartiality, one has their strengths.
	\item [SS]: (24): By samayam on friendliness and other such qualities, the power to transmit them is obtained.
	\item [SP]: (24) By making samyama on friendliness, compassion, etc., one develops the powers of these qualities.
	\item [SV]: (24): By making Samyama on friendship, mercy etc., the yogi excels in the respective qualities.
	\end{itemize}
	
\end{frame}


%%%%%%%%%%%%%%%%%%%%%%%%%%%%%%%%%%%%%%%%%%%%%%%%%%%%%%%%%%%
\begin{frame}[fragile]\frametitle{Samyama}
\begin{sanskrit}
बलेषु हस्तिबलादीनि॥२४॥
\end{sanskrit}

	\begin{itemize}
	\item [HA]: By Practicing Samyama On (Physical) Strength, The Strength Of Elephants Etc. Can Be Acquired.
	\item [IT]: (25): (by performing Samyama) on the strengths (of animals) the strength of an elephant, etc.
	\item [VH]: On strengths – the strength of an elephant, etc.
	\item [BM]: From perfect discipline of the strength of an animal such as an elephant, one gains that strength.
	\item [SS]: (25): By samyama on the strength of elephants and other such animals, their strength is obtained.
	\item [SP]: (25) By making samyama on any kind of strength, such as that of the elephant, one obtains that strength.
	\item [SV]: (25): By making Samyama on the strength of the elephant, etc., that strength comes to the Yogi. 
	\end{itemize}

\end{frame}

%%%%%%%%%%%%%%%%%%%%%%%%%%%%%%%%%%%%%%%%%%%%%%%%%%%%%%%%%%%
\begin{frame}[fragile]\frametitle{Samyama}
\begin{sanskrit}
प्रवृत्त्यालोकन्यासात् सूक्ष्मव्यवहितविप्रकृष्टज्ञानम्॥२५॥
\end{sanskrit}

	\begin{itemize}
	\item [HA]: By Applying The Effulgeant Light Of Higher Sense-Perception (Jyotismati) Knowledge Of Subtle Objects, Or Things Obstructed From View, Or Placed At A Great Distance, Can Be Acquired.
	\item [IT]: (26): Knowledge of the small, the hidden or the distant by directing the light of superphysical faculty.
	\item [VH]: By projecting the brilliance of the pravrtti-finer activity (of citta), knowledge of the subtle, concealed, and distant.
	\item [BM]: From placing light on the minds activity, one has knowledge of that which is subtle, hidden, and distant.
	\item [SS]: (26): By samyama on the Light within, the knowledge of the subtle, hidden and remotes is obtained. [Note: subtle as atoms, hidden as treasure, remote as far distanct lands]
	\item [SP]: (26) By making samyama on the Inner Light one obtains knowledge of what is subtle, hidden, or far distant.
	\item [SV]: (26): By making Samyama on that effulgent light comes the knowledge of the fine, the obstructed, and the remote. that strength comes to the Yogi. 
	\end{itemize}
	
\end{frame}

%%%%%%%%%%%%%%%%%%%%%%%%%%%%%%%%%%%%%%%%%%%%%%%%%%%%%%%%%%%
\begin{frame}[fragile]\frametitle{Samyama}
\begin{sanskrit}
भुवनज्ञानं सूर्ये संयमात्॥२६॥
\end{sanskrit}

	\begin{itemize}
	\item [HA]: (By Practicing Samyama) On The Sun (The Point Of Body Known As The Solar Entrance) The Knowledge Of The Cosmic Regions Is Acquired.
	\item [IT]: (27): Knowledge of the Solar system by performing Samyama on the Sun.
	\item [VH]: By samyama on the sun- knowledge of the worlds.
	\item [BM]: From perfect discipline of the sun, one has knowledge of the worlds.
	\item [SS]: (27): By samyama on the sun, knowledge of the entire solar system is obtained.
	\item [SP]: (27) By making samyama on the sun, one gains knowledge of the cosmic spaces.
	\item [SV]: (27): By making Samyama on the sun, (comes) the knowledge of the world. 
	\end{itemize}
\end{frame}

%%%%%%%%%%%%%%%%%%%%%%%%%%%%%%%%%%%%%%%%%%%%%%%%%%%%%%%%%%%
\begin{frame}[fragile]\frametitle{Yog}
\begin{sanskrit}
चन्द्रे ताराव्यूहज्ञानम्॥२७॥
\end{sanskrit}

	\begin{itemize}
	\item [HA]: (By Practicing Samyama) On The Moon (The Lunar Entrance) Knowledge Of The Arrangements Of The Stars Is Acquired.
	\item [IT]: (28): (By performing Samyama) on the moon knowledge concerning the arrangement of the stars.
	\item [VH]: By samyama on the moon- knowledge of the organization of the stars.
	\item [BM]: From perfect discipline of the moon, one has knowledge of the arrangements of the stars.
	\item [SS]: (28): By samyama on the moon comes knowledge of the stars’ arrangement.
	\item [SP]: (28) By making samyama on the moon, one gains knowledge of the arrangement of the stars.
	\item [SV]: (28): On the moon, (comes) the knowledge of the cluster of stars. 
	\end{itemize}
\end{frame}

%%%%%%%%%%%%%%%%%%%%%%%%%%%%%%%%%%%%%%%%%%%%%%%%%%%%%%%%%%%
\begin{frame}[fragile]\frametitle{Yog}
\begin{sanskrit}
ध्रुवे तद्गतिज्ञानम्॥२८॥
\end{sanskrit}

	\begin{itemize}
	\item [HA]: (By Practicing Samyama) On The Pole Star, Motion Of The Stars Is Known.
	\item [IT]: (29): (By performing Samyama) on the pole-star knowledge of their movements.
	\item [VH]: By samyama on the pole star – knowledge of their motion.
	\item [BM]: From perfect discipline of the polestar, one has knowledge of the movements of the stars.
	\item [SS]: (29): By samyama on the pole star comes knowledge of the stars movements.
	\item [SP]: (29) By making samyama on the polestar, one gains knowledge of the motions of the stars.
	\item [SV]: (29): On the pole star (comes) the knowledge of the motions of the stars. 
	\end{itemize}
\end{frame}


%%%%%%%%%%%%%%%%%%%%%%%%%%%%%%%%%%%%%%%%%%%%%%%%%%%%%%%%%%%
\begin{frame}[fragile]\frametitle{Yog}
\begin{sanskrit}
नाभिचक्रे कायव्यूहज्ञानम्॥२९॥
\end{sanskrit}

	\begin{itemize}
	\item [HA]: (By Practicing Samyama) On The Navel Plexus, Knowledge Of The Composition Of The Body Is Derived.
	\item [IT]: (30): (By performing Samyama) on the navel centre knowledge of the organization of the body.
	\item [VH]: By samyama on the navel cakra – knowledge of the organization of the body.
	\item [BM]: From perfect discipline of the circle of the navel, one has knowledge of the body’s arrangement.
	\item [SS]: (30): By samyama on the navel plexus, knowledge of the body’s constitution is obtained.
	\item [SP]: (30) By making samyama on the navel, one gains knowledge of the constitution of the body.
	\item [SV]: (30): On the navel circle (comes) the knowledge of the constitution of the body. 
	\end{itemize}
\end{frame}

%%%%%%%%%%%%%%%%%%%%%%%%%%%%%%%%%%%%%%%%%%%%%%%%%%%%%%%%%%%
\begin{frame}[fragile]\frametitle{Yog}
\begin{sanskrit}
कण्ठकूपे क्षुत्पिपासानिवृत्तिः॥३०॥
\end{sanskrit}

	\begin{itemize}
	\item [HA]: (By Practicing Samyama) On The Trachea, Hunger And Thirst Can Be Subdued.
	\item [IT]: (31): (By performing Samyama) on the gullet the cessation of hunger and thirst
	\item [VH]: By samyama on the well of the throat – the ceasing of hunger and thirst.
	\item [BM]: From perfect discipline of the cavity of the throat, hunger and thirst are subdued.
	\item [SS]: (31): By samyama on the pit of the throat, cessation of hunger and thirst is achieved.
	\item [SP]: (31) By making samyama on the hollow of the throat, one stills hunger and thirst.
	\item [SV]: (31): On the hollow of the throat (comes) cessation of hunger. 
	\end{itemize}
\end{frame}


%%%%%%%%%%%%%%%%%%%%%%%%%%%%%%%%%%%%%%%%%%%%%%%%%%%%%%%%%%%
\begin{frame}[fragile]\frametitle{Yog}
\begin{sanskrit}
कूर्मनाड्यां स्थैर्यम्॥३१॥
\end{sanskrit}

	\begin{itemize}
	\item [HA]: Calmness Is Attained By Samyama On The Bronchial Tube.
	\item [IT]: (32): (By performing Samyama) on the Kurma-nadi steadiness.
	\item [VH]: On the tortoise duct (tortoise), steadiness.
	\item [BM]: From perfect discipline of the “tortoise vein,” one’s being becomes steady.
	\item [SS]: (32): By samyama on the kurma nadi (a subtle tortoise-shaped tube located below the throat), motionless in the meditative posture is achieved.
	\item [SP]: (32) By making samyama on the tube within the chest, one acquires absolute motionlessness.
	\item [SV]: (32): On the nerve called Kurma (comes) fixity of the body. 
	\end{itemize}
\end{frame}


%%%%%%%%%%%%%%%%%%%%%%%%%%%%%%%%%%%%%%%%%%%%%%%%%%%%%%%%%%%
\begin{frame}[fragile]\frametitle{Yog}
\begin{sanskrit}
मूर्धज्योतिषि सिद्धदर्शनम्॥३२॥
\end{sanskrit}

	\begin{itemize}
	\item [HA]: (By Practicing Samyama) On The Coronal Light, Siddhas Can Be Seen.
	\item [IT]: (33): (By performing Samyama) on the light under the crown of the head vision of perfected Beings.
	\item [VH]: On the light on the top of the head – vision of the perfected ones.
	\item [BM]: From perfect discipline of the light in the head, one gets a vision of the perfected beings.
	\item [SS]: (33): By samyama on the light at the crown of the head (sahasrara chakra), visions of masters and adepts are obtained.
	\item [SP]: (33) By making samyama on the radiance within the back of the head, one becomes able to see the celestial beings.
	\item [SV]: (33): On the light emanating from the top of the head sight of the Siddhas. 
	\end{itemize}
\end{frame}

%%%%%%%%%%%%%%%%%%%%%%%%%%%%%%%%%%%%%%%%%%%%%%%%%%%%%%%%%%%
\begin{frame}[fragile]\frametitle{Yog}
\begin{sanskrit}
प्रातिभाद्वा सर्वम्॥३३॥
\end{sanskrit}

	\begin{itemize}
	\item [HA]: From Knowledge Kown As Pratibha (Intuition) Everything Becomes Known.
	\item [IT]: (34): (Knowledge of) everything from intuition.
	\item [VH]: From pratibha- the flash of illumination, all knowledge.
	\item [BM]: From intuition, one knows everything.
	\item [SS]: (34): Or, in the knowledge that dawns spontaneous enlightenment [through a life of purity], all the powers comes by themselves.
	\item [SP]: (34) All these powers of knowledge may also come to one whose mind is spontaneously enlightened through purity.
	\item [SV]: (34): Or by the power of Pratibha all knowledge. 
	\end{itemize}
\end{frame}

%%%%%%%%%%%%%%%%%%%%%%%%%%%%%%%%%%%%%%%%%%%%%%%%%%%%%%%%%%%
\begin{frame}[fragile]\frametitle{Yog}
\begin{sanskrit}
हृदये चित्तसंवित्॥३४॥
\end{sanskrit}

	\begin{itemize}
	\item [HA]: (By Practicing Samyama) On The Heart, Knowledge Of The Mind Is Acquired.
	\item [IT]: (35): (By performing Samyama) on the heart, awareness of the nature of the mind.
	\item [VH]: On the heart – full knowledge of the citta- the field.
	\item [BM]: From perfect discipline of the heart, one has full consciousness of one’s thought.
	\item [SS]: (35): By samyama on the heart, the knowledge of the mind-stuff is obtained.
	\item [SP]: (35) By making samayama on the heart, one gains knowledge of the contents of the mind.
	\item [SV]: (35): In the heart, knowledge of minds. 
	\end{itemize}
\end{frame}


%%%%%%%%%%%%%%%%%%%%%%%%%%%%%%%%%%%%%%%%%%%%%%%%%%%%%%%%%%%
\begin{frame}[fragile]\frametitle{Yog}
\begin{sanskrit}
सत्त्वपुरुषयोरत्यन्तासंकीर्णयोः प्रत्ययाविशेषो भोगः परार्थान्यस्वार्थसंयमात् पुरुषज्ञानम्॥३५॥
\end{sanskrit}

	\begin{itemize}
	\item [HA]: Experience (Of Pleasure Or Pain) Arises From A Conception Which Does Not Distinguish Between The Two Extremely Different Entities, Viz. Buddhisattva And Parusa. Such Experience Exists For Another (i.e. Parusa). That Is Why Through Samyama On Parusa (Who Oversees All Experience And Also Their Complete Cessation), A Knowledge Regarding Parusa Is Acquired.
	\item [IT]: (36): Experience is the result of inability to distinguish between the Parusa and the Sattva though they are absolutely distinct. Knowledge of the Parusa results from Damyama on theSelf-interest (of the Parusa) apart from another’s interest.
	\item [VH]: Experience is a pratyaya which does not distinguish sattva (guna of brightness, a primary constituent of matter) and parusa – the self as absolutely unmixed. By sanyama on what exists for its own sake (parusa) distinct from that (sattva) which exists for the other – the knowledge of parusa.
	\item [BM]: Worldly experience is caused by a failure to differentiate between the lucid quality of nature and the spirit. From perfect discipline of the distinction between spirit as the subject of itself and the lucid quality of nature as a dependent object, one gains knowledge of the spirit.
	\item [SS]: (36): The intellect and the Parusha (or Atman) are totally different, the intellect existing for the sake of Parusha, while the Parusha exists for its own sake. Not distinguishing this is the cause of all experience; and by samyama on the distinction, knowledge of the Parusha is gained.
	\item [SP]: (36) The power of enjoyment arises from a failure to discriminate between the Atman and the sattwa guna, which are totally different. The sattwa guna is merely the agent of the Atman, which is independent, existing only for its own sake. By making samyama on the independence of the Atman, one gains knowledge of the Atman.
	\item [SV]: (36): Enjoyment comes by the non-discrimination of the very distant soul and Sattva. Its actions are for another; Samyama on this gives knowledge of the Puruca. 
	\end{itemize}
\end{frame}


%%%%%%%%%%%%%%%%%%%%%%%%%%%%%%%%%%%%%%%%%%%%%%%%%%%%%%%%%%%
\begin{frame}[fragile]\frametitle{Yog}
\begin{sanskrit}
ततः प्रातिभश्रावणवेदनादर्शास्वादवार्ता जायन्ते॥३६॥
\end{sanskrit}

	\begin{itemize}
	\item [HA]: Thence (From Knowledge Of Parusa) Arises Pratibha (Prescience), Sravna (Supernormal Power Of Hearing), Vedana (Supernormal Power of Touch), Adarsha (Supernormal Power Of Sight), Asvada (Supernormal Power of Taste) And Varta (Supernormal Power of Smell).
	\item [IT]: (37): Thence are produced intuitional hearing, touch, taste, and smell.
	\item [VH]: From that arises pratibha- the flash of illumination, suprasensory hearing, feeling, seeing, tasting smelling and intelligence.
	\item [BM]: This knowledge engenders intuitive forms of hearing, touch, sight, taste, and smell.
	\item [SS]: (37): From this knowledge arises superphysical hearing, touching, seeing, tasting, and smelling through spontaneous intuition.
	\item [SP]: (37) Hence one gains the knowledge due to spontaneous enlightenment, and obtains supernatural powers of hearing, touch, sight, taste and smell.
	\item [SV]: (37): From that arises the knowledge of hearing, touching, seeing, tasting, and smelling, belonging to Pratibha. 
	\end{itemize}
\end{frame}

%%%%%%%%%%%%%%%%%%%%%%%%%%%%%%%%%%%%%%%%%%%%%%%%%%%%%%%%%%%
\begin{frame}[fragile]\frametitle{Yog}
\begin{sanskrit}
ते समाधावुपसर्गा व्युत्थाने सिद्धयः॥३७॥
\end{sanskrit}
	\begin{itemize}
	\item Yog 
	\end{itemize}
\end{frame}


%%%%%%%%%%%%%%%%%%%%%%%%%%%%%%%%%%%%%%%%%%%%%%%%%%%%%%%%%%%
\begin{frame}[fragile]\frametitle{Yog}
\begin{sanskrit}
बन्धकारणशैथिल्यात्प्रचारसंवेदनाच्च चित्तस्य परशरीरावेशः॥३८॥
\end{sanskrit}
	\begin{itemize}
	\item Yog 
	\end{itemize}
\end{frame}


%%%%%%%%%%%%%%%%%%%%%%%%%%%%%%%%%%%%%%%%%%%%%%%%%%%%%%%%%%%
\begin{frame}[fragile]\frametitle{Yog}
\begin{sanskrit}
उदानजयाज्जलपङ्ककण्टकादिष्वसङ्ग उत्क्रान्तिश्च॥३९॥
\end{sanskrit}
	\begin{itemize}
	\item Yog 
	\end{itemize}
\end{frame}

%%%%%%%%%%%%%%%%%%%%%%%%%%%%%%%%%%%%%%%%%%%%%%%%%%%%%%%%%%%
\begin{frame}[fragile]\frametitle{Yog}
\begin{sanskrit}
समानजयाज्ज्वलनम्॥४०॥
\end{sanskrit}
	\begin{itemize}
	\item Yog 
	\end{itemize}
\end{frame}

%%%%%%%%%%%%%%%%%%%%%%%%%%%%%%%%%%%%%%%%%%%%%%%%%%%%%%%%%%%
\begin{frame}[fragile]\frametitle{Yog}
\begin{sanskrit}
श्रोत्राकाशयोः संबन्धसंयमाद्दिव्यं श्रोत्रम्॥४१॥
\end{sanskrit}
	\begin{itemize}
	\item Yog 
	\end{itemize}
\end{frame}


%%%%%%%%%%%%%%%%%%%%%%%%%%%%%%%%%%%%%%%%%%%%%%%%%%%%%%%%%%%
\begin{frame}[fragile]\frametitle{Yog}
\begin{sanskrit}
कायाकाशयोः संबन्धसंयमाल्लघुतूलसमापत्तेश्चाकाशगमनम्॥४२॥
\end{sanskrit}
	\begin{itemize}
	\item Yog 
	\end{itemize}
\end{frame}

%%%%%%%%%%%%%%%%%%%%%%%%%%%%%%%%%%%%%%%%%%%%%%%%%%%%%%%%%%%
\begin{frame}[fragile]\frametitle{Yog}
\begin{sanskrit}
बहिरकल्पिता वृत्तिर्महाविदेहा ततः प्रकाशावरणक्षयः॥४३॥
\end{sanskrit}
	\begin{itemize}
	\item Yog 
	\end{itemize}
\end{frame}

%%%%%%%%%%%%%%%%%%%%%%%%%%%%%%%%%%%%%%%%%%%%%%%%%%%%%%%%%%%
\begin{frame}[fragile]\frametitle{Yog}
\begin{sanskrit}
बहिरकल्पिता वृत्तिर्महाविदेहा ततः प्रकाशावरणक्षयः॥४३॥
\end{sanskrit}
	\begin{itemize}
	\item Yog 
	\end{itemize}
\end{frame}


%%%%%%%%%%%%%%%%%%%%%%%%%%%%%%%%%%%%%%%%%%%%%%%%%%%%%%%%%%%
\begin{frame}[fragile]\frametitle{Yog}
\begin{sanskrit}
स्थूलस्वरूपसूक्ष्मान्वयार्थवत्त्वसंयमाद भूतजयः॥४४॥
\end{sanskrit}
	\begin{itemize}
	\item Yog 
	\end{itemize}
\end{frame}

%%%%%%%%%%%%%%%%%%%%%%%%%%%%%%%%%%%%%%%%%%%%%%%%%%%%%%%%%%%
\begin{frame}[fragile]\frametitle{Yog}
\begin{sanskrit}
ततोऽणिमादिप्रादुर्भावः कायसंपत्तद्धर्मानभिघातश्च॥४५॥
\end{sanskrit}
	\begin{itemize}
	\item Yog 
	\end{itemize}
\end{frame}


%%%%%%%%%%%%%%%%%%%%%%%%%%%%%%%%%%%%%%%%%%%%%%%%%%%%%%%%%%%
\begin{frame}[fragile]\frametitle{Yog}
\begin{sanskrit}
रूपलावण्यबलवज्रसंहननत्वानि कायसंपत्॥४६॥
\end{sanskrit}
	\begin{itemize}
	\item Yog 
	\end{itemize}
\end{frame}



%%%%%%%%%%%%%%%%%%%%%%%%%%%%%%%%%%%%%%%%%%%%%%%%%%%%%%%%%%%
\begin{frame}[fragile]\frametitle{Yog}
\begin{sanskrit}
ग्रहणस्वरूपास्मितान्वयार्थवत्त्वसंयमादिन्द्रियजयः॥४७॥
\end{sanskrit}
	\begin{itemize}
	\item Yog 
	\end{itemize}
\end{frame}



%%%%%%%%%%%%%%%%%%%%%%%%%%%%%%%%%%%%%%%%%%%%%%%%%%%%%%%%%%%
\begin{frame}[fragile]\frametitle{Yog}
\begin{sanskrit}
ततो मनोजवित्वं विकरणभावः प्रधानजयश्च॥४८॥
\end{sanskrit}
	\begin{itemize}
	\item Yog 
	\end{itemize}
\end{frame}


%%%%%%%%%%%%%%%%%%%%%%%%%%%%%%%%%%%%%%%%%%%%%%%%%%%%%%%%%%%
\begin{frame}[fragile]\frametitle{Yog}
\begin{sanskrit}
सत्त्वपुरुषान्यताख्यातिमात्रस्य सर्वभावाधिष्ठातृत्वं सर्वज्ञातृत्वं च॥४९॥
\end{sanskrit}
	\begin{itemize}
	\item Yog 
	\end{itemize}
\end{frame}


%%%%%%%%%%%%%%%%%%%%%%%%%%%%%%%%%%%%%%%%%%%%%%%%%%%%%%%%%%%
\begin{frame}[fragile]\frametitle{Yog}
\begin{sanskrit}
तद्वैराग्यादपि दोषबीजक्षये कैवल्यम्॥५०॥
\end{sanskrit}
	\begin{itemize}
	\item Yog 
	\end{itemize}
\end{frame}


%%%%%%%%%%%%%%%%%%%%%%%%%%%%%%%%%%%%%%%%%%%%%%%%%%%%%%%%%%%
\begin{frame}[fragile]\frametitle{Yog}
\begin{sanskrit}
स्थान्युपनिमन्त्रणे सङ्गस्मयाकरणं पुनरनिष्टप्रसङ्गात्॥५१॥
\end{sanskrit}
	\begin{itemize}
	\item Yog 
	\end{itemize}
\end{frame}

%%%%%%%%%%%%%%%%%%%%%%%%%%%%%%%%%%%%%%%%%%%%%%%%%%%%%%%%%%%
\begin{frame}[fragile]\frametitle{Yog}
\begin{sanskrit}
स्थान्युपनिमन्त्रणे सङ्गस्मयाकरणं पुनरनिष्टप्रसङ्गात्॥५१॥
\end{sanskrit}
	\begin{itemize}
	\item Yog 
	\end{itemize}
\end{frame}

%%%%%%%%%%%%%%%%%%%%%%%%%%%%%%%%%%%%%%%%%%%%%%%%%%%%%%%%%%%
\begin{frame}[fragile]\frametitle{Yog}
\begin{sanskrit}
क्षणतत्क्रमयोः संयमाद्विवेकजं ज्ञानम्॥५२॥
\end{sanskrit}
	\begin{itemize}
	\item Yog 
	\end{itemize}
\end{frame}


%%%%%%%%%%%%%%%%%%%%%%%%%%%%%%%%%%%%%%%%%%%%%%%%%%%%%%%%%%%
\begin{frame}[fragile]\frametitle{Yog}
\begin{sanskrit}
जातिलक्षणदेशैरन्यतानवच्छेदात् तुल्ययोस्ततः प्रतिपत्तिः॥५३॥
\end{sanskrit}
	\begin{itemize}
	\item Yog 
	\end{itemize}
\end{frame}


%%%%%%%%%%%%%%%%%%%%%%%%%%%%%%%%%%%%%%%%%%%%%%%%%%%%%%%%%%%
\begin{frame}[fragile]\frametitle{Yog}
\begin{sanskrit}
तारकं सर्वविषयं सर्वथाविषयमक्रमं चेति विवेकजं ज्ञानम्॥५४॥
\end{sanskrit}
	\begin{itemize}
	\item Yog 
	\end{itemize}
\end{frame}

%%%%%%%%%%%%%%%%%%%%%%%%%%%%%%%%%%%%%%%%%%%%%%%%%%%%%%%%%%%
\begin{frame}[fragile]\frametitle{Yog}
\begin{sanskrit}
सत्त्वपुरुषयोः शुद्धिसाम्ये कैवल्यमिति॥५५॥
\end{sanskrit}
	\begin{itemize}
	\item Yog 
	\end{itemize}
\end{frame}

