%%%%%%%%%%%%%%%%%%%%%%%%%%%%%%%%%%%%%%%%%%%%%%%%%%%%%%%%%%%%%%%%%%%%%%%%%%%%%%%%%%
\begin{frame}[fragile]\frametitle{}
\begin{center}
{\Large Vibhuti Pada विभूतिपाद}
\end{center}
\end{frame}


%%%%%%%%%%%%%%%%%%%%%%%%%%%%%%%%%%%%%%%%%%%%%%%%%%%%%%%%%%%
\begin{frame}[fragile]\frametitle{Introduction}


	\begin{itemize}
	\item The third chapter of the Patanjali Yoga Sutras is about the results, power, and manifestation once the union is achieved. 
		\item It is said, yogis achieve mystical powers (siddhi) due to the regular practice of yoga. However, this chapter notifies yogis that these very same powers can become a hindrance in their path to liberation. 
			\item Furthermore, it warns against the temptations of the eight siddhis or supernatural powers that a yogi can achieve in the higher levels of spiritual development.
	\item Dives deeper into the last three limbs of yoga, which are collectively known as Samyama (संयम).
	\item Patanjali explains how Samyama is used as the finer tool to remove the subtler veils of ignorance, in this chapter.
	\end{itemize}

\tiny{(Ref: Basic Introduction of Patanjali Yoga Sutras – The Best Knowledge for Yogis - Yoga Moha)}

\end{frame}
