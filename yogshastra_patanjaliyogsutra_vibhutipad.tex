%%%%%%%%%%%%%%%%%%%%%%%%%%%%%%%%%%%%%%%%%%%%%%%%%%%%%%%%%%%%%%%%%%%%%%%%%%%%%%%%%%
\begin{frame}[fragile]\frametitle{}
\begin{center}
{\Large Vibhuti Pada विभूतिपाद}
\end{center}
\end{frame}


%%%%%%%%%%%%%%%%%%%%%%%%%%%%%%%%%%%%%%%%%%%%%%%%%%%%%%%%%%%
\begin{frame}[fragile]\frametitle{Introduction}


	\begin{itemize}
	\item The third chapter of the Patanjali Yoga Sutras is about the results, power, and manifestation once the union is achieved. 
		\item It is said, yogis achieve mystical powers (siddhi) due to the regular practice of yoga. However, this chapter notifies yogis that these very same powers can become a hindrance in their path to liberation. 
			\item Furthermore, it warns against the temptations of the eight siddhis or supernatural powers that a yogi can achieve in the higher levels of spiritual development.
	\item Dives deeper into the last three limbs of yoga, which are collectively known as Samyama (संयम).
	\item Patanjali explains how Samyama is used as the finer tool to remove the subtler veils of ignorance, in this chapter.
	\end{itemize}

\tiny{(Ref: Basic Introduction of Patanjali Yoga Sutras – The Best Knowledge for Yogis - Yoga Moha)}

\end{frame}


%%%%%%%%%%%%%%%%%%%%%%%%%%%%%%%%%%%%%%%%%%%%%%%%%%%%%%%%%%%
\begin{frame}[fragile]\frametitle{Yog}
\begin{sanskrit}
देशबन्धश्चित्तस्य धारणा॥१॥
\end{sanskrit}

	\begin{itemize}
	\item Yog 
	\end{itemize}
\end{frame}

%%%%%%%%%%%%%%%%%%%%%%%%%%%%%%%%%%%%%%%%%%%%%%%%%%%%%%%%%%%
\begin{frame}[fragile]\frametitle{Yog}
\begin{sanskrit}
तत्र प्रत्ययैकतानता ध्यानम्॥२॥
\end{sanskrit}
	\begin{itemize}
	\item Yog 
	\end{itemize}
\end{frame}



%%%%%%%%%%%%%%%%%%%%%%%%%%%%%%%%%%%%%%%%%%%%%%%%%%%%%%%%%%%
\begin{frame}[fragile]\frametitle{Yog}
\begin{sanskrit}
तदेवार्थमात्रनिर्भासं स्वरूपशून्यमिव समाधिः॥३॥
\end{sanskrit}
	\begin{itemize}
	\item Yog 
	\end{itemize}
\end{frame}


%%%%%%%%%%%%%%%%%%%%%%%%%%%%%%%%%%%%%%%%%%%%%%%%%%%%%%%%%%%
\begin{frame}[fragile]\frametitle{Yog}
\begin{sanskrit}
त्रयमेकत्र संयमः॥४॥
\end{sanskrit}
	\begin{itemize}
	\item Yog 
	\end{itemize}
\end{frame}


%%%%%%%%%%%%%%%%%%%%%%%%%%%%%%%%%%%%%%%%%%%%%%%%%%%%%%%%%%%
\begin{frame}[fragile]\frametitle{Yog}
\begin{sanskrit}
तज्जयात्प्रज्ञालोकः॥५॥
\end{sanskrit}
	\begin{itemize}
	\item Yog 
	\end{itemize}
\end{frame}



%%%%%%%%%%%%%%%%%%%%%%%%%%%%%%%%%%%%%%%%%%%%%%%%%%%%%%%%%%%
\begin{frame}[fragile]\frametitle{Yog}
\begin{sanskrit}
तस्य भूमिषु विनियोगः॥६॥
\end{sanskrit}
	\begin{itemize}
	\item Yog 
	\end{itemize}
\end{frame}


%%%%%%%%%%%%%%%%%%%%%%%%%%%%%%%%%%%%%%%%%%%%%%%%%%%%%%%%%%%
\begin{frame}[fragile]\frametitle{Yog}
\begin{sanskrit}
त्रयमन्तरङ्गं पूर्वेभ्यः॥७॥
\end{sanskrit}
	\begin{itemize}
	\item Yog 
	\end{itemize}
\end{frame}


%%%%%%%%%%%%%%%%%%%%%%%%%%%%%%%%%%%%%%%%%%%%%%%%%%%%%%%%%%%
\begin{frame}[fragile]\frametitle{Yog}
\begin{sanskrit}
तदपि बहिरङ्गं निर्बीजस्य॥८॥
\end{sanskrit}
	\begin{itemize}
	\item Yog 
	\end{itemize}
\end{frame}


%%%%%%%%%%%%%%%%%%%%%%%%%%%%%%%%%%%%%%%%%%%%%%%%%%%%%%%%%%%
\begin{frame}[fragile]\frametitle{Yog}
\begin{sanskrit}
व्युत्थाननिरोधसंस्कारयोरभिभवप्रादुर्भावौ निरोधक्षणचित्तान्वयो निरोधपरिणामः॥९॥
\end{sanskrit}
	\begin{itemize}
	\item Yog 
	\end{itemize}
\end{frame}

%%%%%%%%%%%%%%%%%%%%%%%%%%%%%%%%%%%%%%%%%%%%%%%%%%%%%%%%%%%
\begin{frame}[fragile]\frametitle{Yog}
\begin{sanskrit}
तस्य प्रशान्तवाहिता संस्कारात्॥१०॥
\end{sanskrit}
	\begin{itemize}
	\item Yog 
	\end{itemize}
\end{frame}



%%%%%%%%%%%%%%%%%%%%%%%%%%%%%%%%%%%%%%%%%%%%%%%%%%%%%%%%%%%
\begin{frame}[fragile]\frametitle{Yog}
\begin{sanskrit}
सर्वार्थतैकाग्रतयोः क्षयोदयौ चित्तस्य समाधिपरिणामः॥११॥
\end{sanskrit}
	\begin{itemize}
	\item Yog 
	\end{itemize}
\end{frame}


%%%%%%%%%%%%%%%%%%%%%%%%%%%%%%%%%%%%%%%%%%%%%%%%%%%%%%%%%%%
\begin{frame}[fragile]\frametitle{Yog}
\begin{sanskrit}
ततः पुनः शान्तोदितौ तुल्यप्रत्ययौ चित्तस्यैकाग्रतापरिणामः॥१२॥
\end{sanskrit}
	\begin{itemize}
	\item Yog 
	\end{itemize}
\end{frame}


%%%%%%%%%%%%%%%%%%%%%%%%%%%%%%%%%%%%%%%%%%%%%%%%%%%%%%%%%%%
\begin{frame}[fragile]\frametitle{Yog}
\begin{sanskrit}
एतेन भूतेन्द्रियेषु धर्मलक्षणावस्थापरिणामा व्याख्याताः॥१३॥
\end{sanskrit}
	\begin{itemize}
	\item Yog 
	\end{itemize}
\end{frame}

%%%%%%%%%%%%%%%%%%%%%%%%%%%%%%%%%%%%%%%%%%%%%%%%%%%%%%%%%%%
\begin{frame}[fragile]\frametitle{Yog}
\begin{sanskrit}
शान्तोदिताव्यपदेश्यधर्मानुपाती धर्मी॥१४॥
\end{sanskrit}
	\begin{itemize}
	\item Yog 
	\end{itemize}
\end{frame}

%%%%%%%%%%%%%%%%%%%%%%%%%%%%%%%%%%%%%%%%%%%%%%%%%%%%%%%%%%%
\begin{frame}[fragile]\frametitle{Yog}
\begin{sanskrit}
क्रमान्यत्वं परिणामान्यत्वे हेतुः॥१५॥
\end{sanskrit}
	\begin{itemize}
	\item Yog 
	\end{itemize}
\end{frame}


%%%%%%%%%%%%%%%%%%%%%%%%%%%%%%%%%%%%%%%%%%%%%%%%%%%%%%%%%%%
\begin{frame}[fragile]\frametitle{Yog}
\begin{sanskrit}
परिणामत्रयसंयमासदतीतानागतज्ञानम्॥१६॥
\end{sanskrit}
	\begin{itemize}
	\item Yog 
	\end{itemize}
\end{frame}

%%%%%%%%%%%%%%%%%%%%%%%%%%%%%%%%%%%%%%%%%%%%%%%%%%%%%%%%%%%
\begin{frame}[fragile]\frametitle{Yog}
\begin{sanskrit}
शब्दार्थप्रत्ययानामितरेतराध्यासात् संकरस्तत्प्रविभागसंयमात् सर्वभूतरुतज्ञानम्॥१७॥
\end{sanskrit}
	\begin{itemize}
	\item Yog 
	\end{itemize}
\end{frame}

%%%%%%%%%%%%%%%%%%%%%%%%%%%%%%%%%%%%%%%%%%%%%%%%%%%%%%%%%%%
\begin{frame}[fragile]\frametitle{Yog}
\begin{sanskrit}
संस्कारसाक्षत्करणात् पूर्वजातिज्ञानम्॥१८॥
\end{sanskrit}
	\begin{itemize}
	\item Yog 
	\end{itemize}
\end{frame}

%%%%%%%%%%%%%%%%%%%%%%%%%%%%%%%%%%%%%%%%%%%%%%%%%%%%%%%%%%%
\begin{frame}[fragile]\frametitle{Yog}
\begin{sanskrit}
प्रत्ययस्य परचित्तज्ञानम्॥१९॥
\end{sanskrit}
	\begin{itemize}
	\item Yog 
	\end{itemize}
\end{frame}


%%%%%%%%%%%%%%%%%%%%%%%%%%%%%%%%%%%%%%%%%%%%%%%%%%%%%%%%%%%
\begin{frame}[fragile]\frametitle{Yog}
\begin{sanskrit}
न च तत् सालम्बनं तस्याविषयीभूतत्वात्॥२०॥
\end{sanskrit}
	\begin{itemize}
	\item Yog 
	\end{itemize}
\end{frame}

%%%%%%%%%%%%%%%%%%%%%%%%%%%%%%%%%%%%%%%%%%%%%%%%%%%%%%%%%%%
\begin{frame}[fragile]\frametitle{Yog}
\begin{sanskrit}
कायरूपसंयमात् तद्ग्राह्यशक्तिस्तम्भे चक्षुःप्रकाशासंप्रयोगेऽन्तर्धानम्॥२१॥
\end{sanskrit}
	\begin{itemize}
	\item Yog 
	\end{itemize}
\end{frame}

%%%%%%%%%%%%%%%%%%%%%%%%%%%%%%%%%%%%%%%%%%%%%%%%%%%%%%%%%%%
\begin{frame}[fragile]\frametitle{Yog}
\begin{sanskrit}
सोपक्रमं निरुपक्रमं च कर्म तत्संयमादपरान्तज्ञानमरिष्टेभ्यो वा॥२२॥
\end{sanskrit}
	\begin{itemize}
	\item Yog 
	\end{itemize}
\end{frame}

%%%%%%%%%%%%%%%%%%%%%%%%%%%%%%%%%%%%%%%%%%%%%%%%%%%%%%%%%%%
\begin{frame}[fragile]\frametitle{Yog}
\begin{sanskrit}
मैत्र्यादिषु बलानि॥२३॥
\end{sanskrit}
	\begin{itemize}
	\item Yog 
	\end{itemize}
\end{frame}


%%%%%%%%%%%%%%%%%%%%%%%%%%%%%%%%%%%%%%%%%%%%%%%%%%%%%%%%%%%
\begin{frame}[fragile]\frametitle{Yog}
\begin{sanskrit}
बलेषु हस्तिबलादीनि॥२४॥
\end{sanskrit}
	\begin{itemize}
	\item Yog 
	\end{itemize}
\end{frame}

%%%%%%%%%%%%%%%%%%%%%%%%%%%%%%%%%%%%%%%%%%%%%%%%%%%%%%%%%%%
\begin{frame}[fragile]\frametitle{Yog}
\begin{sanskrit}
प्रवृत्त्यालोकन्यासात् सूक्ष्मव्यवहितविप्रकृष्टज्ञानम्॥२५॥
\end{sanskrit}
	\begin{itemize}
	\item Yog 
	\end{itemize}
\end{frame}

%%%%%%%%%%%%%%%%%%%%%%%%%%%%%%%%%%%%%%%%%%%%%%%%%%%%%%%%%%%
\begin{frame}[fragile]\frametitle{Yog}
\begin{sanskrit}
भुवनज्ञानं सूर्ये संयमात्॥२६॥
\end{sanskrit}
	\begin{itemize}
	\item Yog 
	\end{itemize}
\end{frame}

%%%%%%%%%%%%%%%%%%%%%%%%%%%%%%%%%%%%%%%%%%%%%%%%%%%%%%%%%%%
\begin{frame}[fragile]\frametitle{Yog}
\begin{sanskrit}
चन्द्रे ताराव्यूहज्ञानम्॥२७॥
\end{sanskrit}
	\begin{itemize}
	\item Yog 
	\end{itemize}
\end{frame}

%%%%%%%%%%%%%%%%%%%%%%%%%%%%%%%%%%%%%%%%%%%%%%%%%%%%%%%%%%%
\begin{frame}[fragile]\frametitle{Yog}
\begin{sanskrit}
ध्रुवे तद्गतिज्ञानम्॥२८॥
\end{sanskrit}
	\begin{itemize}
	\item Yog 
	\end{itemize}
\end{frame}


%%%%%%%%%%%%%%%%%%%%%%%%%%%%%%%%%%%%%%%%%%%%%%%%%%%%%%%%%%%
\begin{frame}[fragile]\frametitle{Yog}
\begin{sanskrit}
नाभिचक्रे कायव्यूहज्ञानम्॥२९॥
\end{sanskrit}
	\begin{itemize}
	\item Yog 
	\end{itemize}
\end{frame}

%%%%%%%%%%%%%%%%%%%%%%%%%%%%%%%%%%%%%%%%%%%%%%%%%%%%%%%%%%%
\begin{frame}[fragile]\frametitle{Yog}
\begin{sanskrit}
कण्ठकूपे क्षुत्पिपासानिवृत्तिः॥३०॥
\end{sanskrit}
	\begin{itemize}
	\item Yog 
	\end{itemize}
\end{frame}


%%%%%%%%%%%%%%%%%%%%%%%%%%%%%%%%%%%%%%%%%%%%%%%%%%%%%%%%%%%
\begin{frame}[fragile]\frametitle{Yog}
\begin{sanskrit}
कूर्मनाड्यां स्थैर्यम्॥३१॥
\end{sanskrit}
	\begin{itemize}
	\item Yog 
	\end{itemize}
\end{frame}


%%%%%%%%%%%%%%%%%%%%%%%%%%%%%%%%%%%%%%%%%%%%%%%%%%%%%%%%%%%
\begin{frame}[fragile]\frametitle{Yog}
\begin{sanskrit}
मूर्धज्योतिषि सिद्धदर्शनम्॥३२॥
\end{sanskrit}
	\begin{itemize}
	\item Yog 
	\end{itemize}
\end{frame}

%%%%%%%%%%%%%%%%%%%%%%%%%%%%%%%%%%%%%%%%%%%%%%%%%%%%%%%%%%%
\begin{frame}[fragile]\frametitle{Yog}
\begin{sanskrit}
प्रातिभाद्वा सर्वम्॥३३॥
\end{sanskrit}
	\begin{itemize}
	\item Yog 
	\end{itemize}
\end{frame}

%%%%%%%%%%%%%%%%%%%%%%%%%%%%%%%%%%%%%%%%%%%%%%%%%%%%%%%%%%%
\begin{frame}[fragile]\frametitle{Yog}
\begin{sanskrit}
हृदये चित्तसंवित्॥३४॥
\end{sanskrit}
	\begin{itemize}
	\item Yog 
	\end{itemize}
\end{frame}


%%%%%%%%%%%%%%%%%%%%%%%%%%%%%%%%%%%%%%%%%%%%%%%%%%%%%%%%%%%
\begin{frame}[fragile]\frametitle{Yog}
\begin{sanskrit}
सत्त्वपुरुषयोरत्यन्तासंकीर्णयोः प्रत्ययाविशेषो भोगः परार्थान्यस्वार्थसंयमात् पुरुषज्ञानम्॥३५॥
\end{sanskrit}
	\begin{itemize}
	\item Yog 
	\end{itemize}
\end{frame}


%%%%%%%%%%%%%%%%%%%%%%%%%%%%%%%%%%%%%%%%%%%%%%%%%%%%%%%%%%%
\begin{frame}[fragile]\frametitle{Yog}
\begin{sanskrit}
ततः प्रातिभश्रावणवेदनादर्शास्वादवार्ता जायन्ते॥३६॥
\end{sanskrit}
	\begin{itemize}
	\item Yog 
	\end{itemize}
\end{frame}

%%%%%%%%%%%%%%%%%%%%%%%%%%%%%%%%%%%%%%%%%%%%%%%%%%%%%%%%%%%
\begin{frame}[fragile]\frametitle{Yog}
\begin{sanskrit}
ते समाधावुपसर्गा व्युत्थाने सिद्धयः॥३७॥
\end{sanskrit}
	\begin{itemize}
	\item Yog 
	\end{itemize}
\end{frame}


%%%%%%%%%%%%%%%%%%%%%%%%%%%%%%%%%%%%%%%%%%%%%%%%%%%%%%%%%%%
\begin{frame}[fragile]\frametitle{Yog}
\begin{sanskrit}
बन्धकारणशैथिल्यात्प्रचारसंवेदनाच्च चित्तस्य परशरीरावेशः॥३८॥
\end{sanskrit}
	\begin{itemize}
	\item Yog 
	\end{itemize}
\end{frame}


%%%%%%%%%%%%%%%%%%%%%%%%%%%%%%%%%%%%%%%%%%%%%%%%%%%%%%%%%%%
\begin{frame}[fragile]\frametitle{Yog}
\begin{sanskrit}
उदानजयाज्जलपङ्ककण्टकादिष्वसङ्ग उत्क्रान्तिश्च॥३९॥
\end{sanskrit}
	\begin{itemize}
	\item Yog 
	\end{itemize}
\end{frame}

%%%%%%%%%%%%%%%%%%%%%%%%%%%%%%%%%%%%%%%%%%%%%%%%%%%%%%%%%%%
\begin{frame}[fragile]\frametitle{Yog}
\begin{sanskrit}
समानजयाज्ज्वलनम्॥४०॥
\end{sanskrit}
	\begin{itemize}
	\item Yog 
	\end{itemize}
\end{frame}

%%%%%%%%%%%%%%%%%%%%%%%%%%%%%%%%%%%%%%%%%%%%%%%%%%%%%%%%%%%
\begin{frame}[fragile]\frametitle{Yog}
\begin{sanskrit}
श्रोत्राकाशयोः संबन्धसंयमाद्दिव्यं श्रोत्रम्॥४१॥
\end{sanskrit}
	\begin{itemize}
	\item Yog 
	\end{itemize}
\end{frame}


%%%%%%%%%%%%%%%%%%%%%%%%%%%%%%%%%%%%%%%%%%%%%%%%%%%%%%%%%%%
\begin{frame}[fragile]\frametitle{Yog}
\begin{sanskrit}
कायाकाशयोः संबन्धसंयमाल्लघुतूलसमापत्तेश्चाकाशगमनम्॥४२॥
\end{sanskrit}
	\begin{itemize}
	\item Yog 
	\end{itemize}
\end{frame}

%%%%%%%%%%%%%%%%%%%%%%%%%%%%%%%%%%%%%%%%%%%%%%%%%%%%%%%%%%%
\begin{frame}[fragile]\frametitle{Yog}
\begin{sanskrit}
बहिरकल्पिता वृत्तिर्महाविदेहा ततः प्रकाशावरणक्षयः॥४३॥
\end{sanskrit}
	\begin{itemize}
	\item Yog 
	\end{itemize}
\end{frame}

%%%%%%%%%%%%%%%%%%%%%%%%%%%%%%%%%%%%%%%%%%%%%%%%%%%%%%%%%%%
\begin{frame}[fragile]\frametitle{Yog}
\begin{sanskrit}
बहिरकल्पिता वृत्तिर्महाविदेहा ततः प्रकाशावरणक्षयः॥४३॥
\end{sanskrit}
	\begin{itemize}
	\item Yog 
	\end{itemize}
\end{frame}


%%%%%%%%%%%%%%%%%%%%%%%%%%%%%%%%%%%%%%%%%%%%%%%%%%%%%%%%%%%
\begin{frame}[fragile]\frametitle{Yog}
\begin{sanskrit}
स्थूलस्वरूपसूक्ष्मान्वयार्थवत्त्वसंयमाद भूतजयः॥४४॥
\end{sanskrit}
	\begin{itemize}
	\item Yog 
	\end{itemize}
\end{frame}

%%%%%%%%%%%%%%%%%%%%%%%%%%%%%%%%%%%%%%%%%%%%%%%%%%%%%%%%%%%
\begin{frame}[fragile]\frametitle{Yog}
\begin{sanskrit}
ततोऽणिमादिप्रादुर्भावः कायसंपत्तद्धर्मानभिघातश्च॥४५॥
\end{sanskrit}
	\begin{itemize}
	\item Yog 
	\end{itemize}
\end{frame}


%%%%%%%%%%%%%%%%%%%%%%%%%%%%%%%%%%%%%%%%%%%%%%%%%%%%%%%%%%%
\begin{frame}[fragile]\frametitle{Yog}
\begin{sanskrit}
रूपलावण्यबलवज्रसंहननत्वानि कायसंपत्॥४६॥
\end{sanskrit}
	\begin{itemize}
	\item Yog 
	\end{itemize}
\end{frame}



%%%%%%%%%%%%%%%%%%%%%%%%%%%%%%%%%%%%%%%%%%%%%%%%%%%%%%%%%%%
\begin{frame}[fragile]\frametitle{Yog}
\begin{sanskrit}
ग्रहणस्वरूपास्मितान्वयार्थवत्त्वसंयमादिन्द्रियजयः॥४७॥
\end{sanskrit}
	\begin{itemize}
	\item Yog 
	\end{itemize}
\end{frame}



%%%%%%%%%%%%%%%%%%%%%%%%%%%%%%%%%%%%%%%%%%%%%%%%%%%%%%%%%%%
\begin{frame}[fragile]\frametitle{Yog}
\begin{sanskrit}
ततो मनोजवित्वं विकरणभावः प्रधानजयश्च॥४८॥
\end{sanskrit}
	\begin{itemize}
	\item Yog 
	\end{itemize}
\end{frame}


%%%%%%%%%%%%%%%%%%%%%%%%%%%%%%%%%%%%%%%%%%%%%%%%%%%%%%%%%%%
\begin{frame}[fragile]\frametitle{Yog}
\begin{sanskrit}
सत्त्वपुरुषान्यताख्यातिमात्रस्य सर्वभावाधिष्ठातृत्वं सर्वज्ञातृत्वं च॥४९॥
\end{sanskrit}
	\begin{itemize}
	\item Yog 
	\end{itemize}
\end{frame}


%%%%%%%%%%%%%%%%%%%%%%%%%%%%%%%%%%%%%%%%%%%%%%%%%%%%%%%%%%%
\begin{frame}[fragile]\frametitle{Yog}
\begin{sanskrit}
तद्वैराग्यादपि दोषबीजक्षये कैवल्यम्॥५०॥
\end{sanskrit}
	\begin{itemize}
	\item Yog 
	\end{itemize}
\end{frame}


%%%%%%%%%%%%%%%%%%%%%%%%%%%%%%%%%%%%%%%%%%%%%%%%%%%%%%%%%%%
\begin{frame}[fragile]\frametitle{Yog}
\begin{sanskrit}
स्थान्युपनिमन्त्रणे सङ्गस्मयाकरणं पुनरनिष्टप्रसङ्गात्॥५१॥
\end{sanskrit}
	\begin{itemize}
	\item Yog 
	\end{itemize}
\end{frame}

%%%%%%%%%%%%%%%%%%%%%%%%%%%%%%%%%%%%%%%%%%%%%%%%%%%%%%%%%%%
\begin{frame}[fragile]\frametitle{Yog}
\begin{sanskrit}
स्थान्युपनिमन्त्रणे सङ्गस्मयाकरणं पुनरनिष्टप्रसङ्गात्॥५१॥
\end{sanskrit}
	\begin{itemize}
	\item Yog 
	\end{itemize}
\end{frame}

%%%%%%%%%%%%%%%%%%%%%%%%%%%%%%%%%%%%%%%%%%%%%%%%%%%%%%%%%%%
\begin{frame}[fragile]\frametitle{Yog}
\begin{sanskrit}
क्षणतत्क्रमयोः संयमाद्विवेकजं ज्ञानम्॥५२॥
\end{sanskrit}
	\begin{itemize}
	\item Yog 
	\end{itemize}
\end{frame}


%%%%%%%%%%%%%%%%%%%%%%%%%%%%%%%%%%%%%%%%%%%%%%%%%%%%%%%%%%%
\begin{frame}[fragile]\frametitle{Yog}
\begin{sanskrit}
जातिलक्षणदेशैरन्यतानवच्छेदात् तुल्ययोस्ततः प्रतिपत्तिः॥५३॥
\end{sanskrit}
	\begin{itemize}
	\item Yog 
	\end{itemize}
\end{frame}


%%%%%%%%%%%%%%%%%%%%%%%%%%%%%%%%%%%%%%%%%%%%%%%%%%%%%%%%%%%
\begin{frame}[fragile]\frametitle{Yog}
\begin{sanskrit}
तारकं सर्वविषयं सर्वथाविषयमक्रमं चेति विवेकजं ज्ञानम्॥५४॥
\end{sanskrit}
	\begin{itemize}
	\item Yog 
	\end{itemize}
\end{frame}

%%%%%%%%%%%%%%%%%%%%%%%%%%%%%%%%%%%%%%%%%%%%%%%%%%%%%%%%%%%
\begin{frame}[fragile]\frametitle{Yog}
\begin{sanskrit}
सत्त्वपुरुषयोः शुद्धिसाम्ये कैवल्यमिति॥५५॥
\end{sanskrit}
	\begin{itemize}
	\item Yog 
	\end{itemize}
\end{frame}

