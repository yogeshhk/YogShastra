%%%%%%%%%%%%%%%%%%%%%%%%%%%%%%%%%%%%%%%%%%%%%%%%%%%%%%%%%%%%%%%%%%%%%%%%%%%%%%%%%%
\begin{frame}[fragile]\frametitle{}
\begin{center}
{\Large Dhyan ध्यान}
\end{center}
\end{frame}


%%%%%%%%%%%%%%%%%%%%%%%%%%%%%%%%%%%%%%%%%%%%%%%%%%%%%%%%%%%
\begin{frame}[fragile]\frametitle{Introduction}

Tatra Pratyayaikataanataa Dhyanam

तत्र प्रत्ययैकतानता ध्यानम्||

	\begin{itemize}
	\item An  unbroken  flow  of 
knowledge  in  that  object  is 
Dhyanam.  
\item The  mind  tries  to 
think  of  one  object  to  hold 
itself  to  one  point  and  if  the 
mind  succeeds  in  receiving 
the  sensations  only  through 
that  part  or  point    and  if  the 
mind  can  keep  itself  in  that 
state  for  some  time,  it  is 
called Dhyana
	\end{itemize}

\end{frame}



