%%%%%%%%%%%%%%%%%%%%%%%%%%%%%%%%%%%%%%%%%%%%%%%%%%%%%%%%%%%%%%%%%%%%%%%%%%%%%%%%%%
\begin{frame}[fragile]\frametitle{}
\begin{center}
{\Large Introduction to Yog Shastra}
\end{center}
\end{frame}

%%%%%%%%%%%%%%%%%%%%%%%%%%%%%%%%%%%%%%%%%%%%%%%%%%%%%%%%%%%
\begin{frame}[fragile]\frametitle{Yog (योग) == Asana (आसन)?}
When people hear the word ``Yoga''

	\begin{itemize}
	\item You imagine someone doing contorted, nearly impossible postures.
	\item Or some skeleton siting aloof with long hair, beard
	\item Or anyone earing full saffron robe
	\end{itemize}

\end{frame}

%%%%%%%%%%%%%%%%%%%%%%%%%%%%%%%%%%%%%%%%%%%%%%%%%%%%%%%%%%%
\begin{frame}[fragile]\frametitle{Yog (योग) == Asana (आसन)?}

	\begin{itemize}
	\item Well, these could be some, but rather less important aspects of Yog
	\item Yog (योग, not Yoga ie योगा) is more about mind than getting your leg behind the neck and still keeping the face straight!!
	\item ``Patanjali Yog Sutra (पतञ्जलि योग सूत्र)'' is the most concise, complete and authoritative work on Yog.
	\item Out of 195 sutras, only 3 discuss Yog postures or asanas (आसन)
	\item And he says, ``Sthir Sukham Asanam (स्थिर सुखम आसनम्)'', Postures should be steady and pleasant (2:46), a balance of efforts and ease.
	\end{itemize}

\end{frame}


%%%%%%%%%%%%%%%%%%%%%%%%%%%%%%%%%%%%%%%%%%%%%%%%%%%%%%%%%%%
\begin{frame}[fragile]\frametitle{What is Yog?}

Yog (योग) is derived from verb-root yuj (युज्) means to unite

Yog is a path that leads to union of:
	\begin{itemize}
	\item Physical body
	\item Mental body
	\item Psychological body
	\item Spiritual body
	\end{itemize}

\end{frame}

%%%%%%%%%%%%%%%%%%%%%%%%%%%%%%%%%%%%%%%%%%%%%%%%%%%%%%%%%%%
\begin{frame}[fragile]\frametitle{Definitions of Yog?}

	\begin{itemize}
	\item \textbf{Patanjali}: yogas-chitta-vrutti-norodhah (योगस्चित्तवृत्तिनिरोध:) (YogSutra 1:2): Yoga is the cessation of modifications of the mind
	\item \textbf{Yoga-Vasishta}: manah prashamanah upayah yog itibhidyate (मन: प्रशमन: उपाय: योग इत्याभियते) (Yog Vashishth 3:9:32): Yoga is a skillful way to calm down the mind
	\item \textbf{Bhagavad Gita}: yogah karmasu kaushalam (योग: कर्मसु कौशलम्:) (Bhagvad Gita 2:50): Yoga is dexterity in action
	\item \textbf{Bhagavad Gita}: samatvam yog uchyate (समत्वं योग उच्यते) (Bhagvad Gita ?:??): Equanimity towards everything is Yog.	
	\end{itemize}

\end{frame}

%%%%%%%%%%%%%%%%%%%%%%%%%%%%%%%%%%%%%%%%%%%%%%%%%%%%%%%%%%%
\begin{frame}[fragile]\frametitle{Contradictions in Definitions of Yog}

	\begin{itemize}
	\item There were two systems of Yog. One had gotten lost. Krishna is trying to revive the lost one, in Gita श्रीमद्भगवद्गीता. This goes for DOING ACTION
	\item The more known one is the Patanjali Yog. This goes for STOPPING ACTION.
	\item Patanjali type of Yog is mentioned in Upanishads also. Go to secluded place, in the forest, in the cave, by the stream, etc. (Ref: Mokshadhrama मोक्षधर्म)
	\item Even Arjun अर्जुन had this understanding. He gets confused. Why do the action/battle?
	\item Krishna does not reject renunciation type of yoga, but keeps it aside (Raj राजयोग / dhyan ध्यानयोग in gita itself) and says its not for you (Arjun), you need Action Yoga. (karma कर्मयोग)
	\item Karma becomes yoga when only you have no attachment to it. Doing the ACTION as an offering to the Lord (Krishna कृष्ण)
	\item Even Patanjali Yog is a practice. Its Asana आसन aspect is fully action oriented. But thats just to prepare you for the final goal of Samadhi (समाधि)
	\end{itemize}

\tiny{(Ref: Edwin Bryant sutras - Kristen Vrabel)}


\end{frame}


%%%%%%%%%%%%%%%%%%%%%%%%%%%%%%%%%%%%%%%%%%%%%%%%%%%%%%%%%%%
\begin{frame}[fragile]\frametitle{Why Yog?}

	\begin{itemize}
	\item State of cessation of misery is called as moksh (मोक्ष), kaivalya (कैवल्य), nishreyas (निश्रेयस), parampad (परमपद).
	\item One can attain this state only when we have clear understanding (sukshma gyan सुक्ष्म ज्ञान) of world around.
	\item For this understanding, one needs specific state of mind (chitta vishuddhi चित्त विशुद्धि)
	\item To attain this state of mind, sages have advised many ways, and one of the ways is Yog (Yog dasrhan योगदर्शन)
	\item Yogsharta by Patanjali has given ways of yog darshan to attain moksh.
	\end{itemize}

\tiny{(Ref: पातंजल योग सूत्र | Yog Darshan - Yoga And Ayurveda Science Youtube channel)}

\end{frame}

%%%%%%%%%%%%%%%%%%%%%%%%%%%%%%%%%%%%%%%%%%%%%%%%%%%%%%%%%%%
\begin{frame}[fragile]\frametitle{ योग दर्शन Yog Darshan}

	\begin{itemize}
	\item Removal of human sufferings through practice of yoga
	\item Explains purpose of life, reincarnation
	\item Experience of ultimate reality - self realization (kaivayla कैवल्य )
	\end{itemize}

\tiny{(Ref: Patanjali Yoga Sutra Dr Mrudula Chaudhari)}

\end{frame}

